\documentclass{article}

\usepackage{fancyhdr}
\usepackage{extramarks}
\usepackage{amsmath}
\usepackage{amsthm}
\usepackage{amsfonts}
\usepackage{empheq}
\usepackage{mathtools}
\usepackage{physics}

%
% Basic Document Settings
%

\topmargin=-0.45in
\evensidemargin=0in
\oddsidemargin=0in
\textwidth=6.5in
\textheight=9.0in
\headsep=0.25in

\linespread{1.2}

\pagestyle{fancy}
\lhead{\hmwkAuthorName}
\chead{\hmwkClass: \hmwkTitle}
\rhead{\firstxmark}
%\lfoot{\lastxmark}
\cfoot{\thepage}

\renewcommand\headrulewidth{0.4pt}
\renewcommand\footrulewidth{0.4pt}

\setlength\parindent{0pt}

%
% Create Question Sections
%

\newcommand{\enterQuestionHeader}[1]{
	\nobreak\extramarks{}{Question \arabic{#1} cont'd on next page\ldots}\nobreak{}
	\nobreak\extramarks{Question \arabic{#1} (cont'd)}{Question \arabic{#1} cont'd on next page\ldots}\nobreak{}
}

\newcommand{\exitQuestionHeader}[1]{
	\nobreak\extramarks{Question \arabic{#1} (cont'd)}{Question \arabic{#1} cont'd on next page\ldots}\nobreak{}
	\stepcounter{#1}
	\nobreak\extramarks{Question \arabic{#1}}{}\nobreak{}
}

\setcounter{secnumdepth}{0}
\newcounter{partCounter}
\newcounter{homeworkQuestionCounter}
\setcounter{homeworkQuestionCounter}{1}
\nobreak\extramarks{Question \arabic{homeworkQuestionCounter}}{}\nobreak{}

%
% Homework Question Environment
%
% This environment takes an optional argument. When given, it will adjust the
% problem counter. This is useful for when the problems given for your
% assignment aren't sequential. See the last 3 problems of this template for an
% example.
%
\newenvironment{homeworkQuestion}[1][-1]{
	\ifnum#1>0
	\setcounter{homeworkQuestionCounter}{#1}
	\fi
	\section{Question \arabic{homeworkQuestionCounter}}
	\rule{0.7\textwidth}{3pt}
	\setcounter{partCounter}{1}
	\enterQuestionHeader{homeworkQuestionCounter}
}{
	\exitQuestionHeader{homeworkQuestionCounter}
}

%
% Homework Details
%   - Title
%   - Due date
%   - Class
%   - Section/Time
%   - Instructor
%   - Author
%

\newcommand{\hmwkTitle}{HW\ \#2}
\newcommand{\hmwkDueDate}{}
\newcommand{\hmwkClass}{CpE 520}
\newcommand{\institute}{West Virginia University}
\newcommand{\hmwkClassInstructor}{Professor Nasser Nasrabadi}
\newcommand{\hmwkAuthorName}{\textbf{Ali Zafari}}

%
% Title Page
%

\title{
	\vspace{2in}
	\textmd{\textbf{\hmwkClass:\ \hmwkTitle}}\\
	\vspace{0.1in}\large{\institute}\\
%	\vspace{0.1in}\large{\textit{\hmwkClassInstructor}}
	\vspace{3in}
}

\author{\hmwkAuthorName}
\date{}

\renewcommand{\part}[1]{\textbf{\huge Part \Alph{partCounter}}\stepcounter{partCounter}\\}


\begin{document}
	
	\pagenumbering{gobble}% prevent cover page of numbering
	\maketitle
	\pagebreak % let cover page free to the end of page
	\pagenumbering{arabic} % start page numbering again from 1 and print them!
	
	\begin{homeworkQuestion}
		\subsection{Part (a): Inverse of $\mathbf{y_{n\times1}=A_{n\times m}x_{m\times 1}}$ where n<m and rank=n }
		In this case there are more unknowns than equations so the system is underdetermined. Because the matrix $A$ is full rank with $rank=n$, $AA^T$ will be an $n\times n$ non-singular square matrix. Therefore, there will be an inverse for it like below:
		\begin{equation*}
		(AA^T)(AA^T)^{-1}  = I
		\tag{*}
		\end{equation*}
		As it is clear from the above equation, we may use $A^T(AA^T)^{-1}$ as a right inverse for matrix $A$. We first assume that the inverse we are seeking for has an equation like  $\mathbf{x_{m\times 1}=R_{m\times n}y_{n\times1}}$. Finding $R_{m\times n}$ will be the goal. We propose $R_{m\times n} = A^T(AA^T)^{-1}$ and through the equations below we will check if it will result in our main equation:
		\begin{empheq}[box=\fbox]{equation*}
		\mathbf{x}\mathbf{=A^T(AA^T)^{-1}y}
		\end{empheq}
		\begin{align*}
		\mathbf{x}&\mathbf{=A^T(AA^T)^{-1}y}\\
		\xrightarrow{\times A}\mathbf{Ax}&=\mathbf{AA^T(AA^T)^{-1}y}\\
		\xrightarrow{\text{using *}}\mathbf{Ax}&=\mathbf{y}
		\end{align*}
		\rule{0.2\textwidth}{0.5pt}
		
		\subsection{Part (b): Inverse of $\mathbf{y_{n\times1}=A_{n\times m}x_{m\times 1}}$ where n>m and rank=m}
		In this case there are more equations than unknowns, so the system is overdetermined. Because the matrix $A$ is full rank with $rank=m$, $A^TA$ will be an $m\times m$ non-singular square matrix. Therefore, there will be an inverse for it like below:
		\begin{equation*}
		(A^TA)^{-1}(A^TA)  = I
		\tag{**}
		\end{equation*}
		As it is clear from the above equation, we may use $(A^TA)^{-1}A^T$ as a left inverse for matrix $A$. 
		\begin{align*}
		\mathbf{Ax}&\mathbf{=y}\\
		\xrightarrow{\times A^T}\mathbf{A^TAx}&\mathbf{=A^Ty}\\
		\xrightarrow[\text{using *}]{\times (A^TA)^{-1}}\;\;\;\;\;\;\;\;\;\mathbf{x}&\mathbf{=(A^TA)^{-1}A^Ty}
		\end{align*}
		\begin{empheq}[box=\fbox]{equation*}
		\mathbf{x}\mathbf{=(A^TA)^{-1}A^Ty}
		\end{empheq}
	\end{homeworkQuestion}
	\pagebreak
	
	\begin{homeworkQuestion}
		\subsection{Part (a): Minimizing $\displaystyle \mathbf{\tilde{x}=\min_{x} \{\norm{x}_2^2+\frac{1}{2}\lambda^T(y-Ax)\} }$\\to find the solution of  $\mathbf{y_{n\times1}=A_{n\times m}x_{m\times 1}}$ where n<m and rank=n }
		(Underdetermined System)\\
		We define the function $J(x,\lambda)$ as:
		\begin{equation*}
		J(x,\lambda)=\norm{x}_2^2+\frac{1}{2}\lambda^T(y-Ax)
		\end{equation*} 
		now we find its partial derivatives and let them be zero:
		\begin{align*}
		\frac{\partial J}{\partial x}&=2x-\frac{1}{2}A^T\lambda=0 \rightarrow x=\frac{1}{4}A^T\lambda \tag{1}\\
		\frac{\partial J}{\partial \lambda}&=\frac{1}{2}(y-Ax)=0 \rightarrow y=Ax \tag{2}
		\end{align*}
		by putting equation (1) into (2):
		\begin{equation*}
		y=\frac{1}{4}AA^T\lambda
		\end{equation*}
		we know that $AA^T$ is invertible so we can multiply both sides by $(AA^T)^{-1}$:
		\begin{equation*}
		\lambda=4(AA^T)^{-1}y
		\end{equation*}
		we use this $\lambda$ to replace it in equation (1):
		\begin{equation*}
		x=\frac{1}{4}A^T(4(AA^T)^{-1}y)
		\end{equation*}
		\begin{empheq}[box=\fbox]{equation*}
		x=A^T(AA^T)^{-1}y
		\end{empheq}
	\rule{0.2\textwidth}{0.5pt}
	
	\subsection{Part (b-1): Minimizing $\displaystyle \mathbf{\tilde{x}=\min_{x} \{(y-Ax)^2\} }$\\to find the solution of  $\mathbf{y_{n\times1}=A_{n\times m}x_{m\times 1}}$ where n>m and rank=m }
	(Overdetermined System)\\
	We define the function $J(x)$ as:
	\begin{align*}
	J(x)=(y-Ax)^2&=\norm{y-Ax}^2_2\\
								&=(y-Ax)^T(y-Ax)\\
								&=y^Ty-2y^TAx+x^TA^TAx
	\end{align*} 
	now we find its partial derivatives and let them be zero:
	\begin{align*}
	\frac{\partial J}{\partial x}=-2A^Ty+2A^TAx=0\\
	A^TAx=A^Ty
	\end{align*}
	we know that $A^TA$ is invertible so we can multiply both sides by $(A^TA)^{-1}$:
	\begin{empheq}[box=\fbox]{equation*}
	x=(A^TA)^{-1}A^Ty
	\end{empheq}
	\rule{0.2\textwidth}{0.5pt}
	
	\subsection{Part(b-2) : Minimizing $\displaystyle \mathbf{\tilde{x}=\min_{x} \{(y-Ax)^2+\frac{1}{2}\lambda\norm{x}_2^2\} }$\\to find the solution of  $\mathbf{y_{n\times1}=A_{n\times m}x_{m\times 1}}$ where n>m and rank=m }
	(Overdetermined System)\\
	We define the function $J(x)$ as:
	\begin{align*}
	J(x)=(y-Ax)^2+\frac{1}{2}\lambda\norm{x}_2^2&=\norm{y-Ax}^2_2+\frac{1}{2}\lambda\norm{x}_2^2\\
	&=(y-Ax)^T(y-Ax)+\frac{1}{2}\lambda x^Tx\\
	&=y^Ty-2y^TAx+x^TA^TAx+\frac{1}{2}\lambda x^Tx
	\end{align*} 
	now we find its partial derivatives and let them be zero:
	\begin{align*}
	\frac{\partial J}{\partial x}=-2A^Ty+2A^TAx+\lambda x&=0\\
	(A^TA+\frac{1}{2}\lambda I)x&=A^Ty
	\end{align*}
	\begin{empheq}[box=\fbox]{equation*}
	x=(A^TA+\frac{1}{2}\lambda I)^{-1}A^Ty
	\end{empheq}
\end{homeworkQuestion}



\end{document}