\documentclass[11pt]{article}

    \usepackage[breakable]{tcolorbox}
    \usepackage{parskip} % Stop auto-indenting (to mimic markdown behaviour)
    
    \usepackage{iftex}
    \ifPDFTeX
    	\usepackage[T1]{fontenc}
    	\usepackage{mathpazo}
    \else
    	\usepackage{fontspec}
    \fi

    % Basic figure setup, for now with no caption control since it's done
    % automatically by Pandoc (which extracts ![](path) syntax from Markdown).
    \usepackage{graphicx}
    % Maintain compatibility with old templates. Remove in nbconvert 6.0
    \let\Oldincludegraphics\includegraphics
    % Ensure that by default, figures have no caption (until we provide a
    % proper Figure object with a Caption API and a way to capture that
    % in the conversion process - todo).
    \usepackage{caption}
    \DeclareCaptionFormat{nocaption}{}
    \captionsetup{format=nocaption,aboveskip=0pt,belowskip=0pt}

    \usepackage[Export]{adjustbox} % Used to constrain images to a maximum size
    \adjustboxset{max size={0.9\linewidth}{0.9\paperheight}}
    \usepackage{float}
    \floatplacement{figure}{H} % forces figures to be placed at the correct location
    \usepackage{xcolor} % Allow colors to be defined
    \usepackage{enumerate} % Needed for markdown enumerations to work
    \usepackage{geometry} % Used to adjust the document margins
    \usepackage{amsmath} % Equations
    \usepackage{amssymb} % Equations
    \usepackage{textcomp} % defines textquotesingle
    % Hack from http://tex.stackexchange.com/a/47451/13684:
    \AtBeginDocument{%
        \def\PYZsq{\textquotesingle}% Upright quotes in Pygmentized code
    }
    \usepackage{upquote} % Upright quotes for verbatim code
    \usepackage{eurosym} % defines \euro
    \usepackage[mathletters]{ucs} % Extended unicode (utf-8) support
    \usepackage{fancyvrb} % verbatim replacement that allows latex
    \usepackage{grffile} % extends the file name processing of package graphics 
                         % to support a larger range
    \makeatletter % fix for grffile with XeLaTeX
    \def\Gread@@xetex#1{%
      \IfFileExists{"\Gin@base".bb}%
      {\Gread@eps{\Gin@base.bb}}%
      {\Gread@@xetex@aux#1}%
    }
    \makeatother

    % The hyperref package gives us a pdf with properly built
    % internal navigation ('pdf bookmarks' for the table of contents,
    % internal cross-reference links, web links for URLs, etc.)
    \usepackage{hyperref}
    % The default LaTeX title has an obnoxious amount of whitespace. By default,
    % titling removes some of it. It also provides customization options.
    \usepackage{titling}
    \usepackage{longtable} % longtable support required by pandoc >1.10
    \usepackage{booktabs}  % table support for pandoc > 1.12.2
    \usepackage[inline]{enumitem} % IRkernel/repr support (it uses the enumerate* environment)
    \usepackage[normalem]{ulem} % ulem is needed to support strikethroughs (\sout)
                                % normalem makes italics be italics, not underlines
    \usepackage{mathrsfs}
    

    
    % Colors for the hyperref package
    \definecolor{urlcolor}{rgb}{0,.145,.698}
    \definecolor{linkcolor}{rgb}{.71,0.21,0.01}
    \definecolor{citecolor}{rgb}{.12,.54,.11}

    % ANSI colors
    \definecolor{ansi-black}{HTML}{3E424D}
    \definecolor{ansi-black-intense}{HTML}{282C36}
    \definecolor{ansi-red}{HTML}{E75C58}
    \definecolor{ansi-red-intense}{HTML}{B22B31}
    \definecolor{ansi-green}{HTML}{00A250}
    \definecolor{ansi-green-intense}{HTML}{007427}
    \definecolor{ansi-yellow}{HTML}{DDB62B}
    \definecolor{ansi-yellow-intense}{HTML}{B27D12}
    \definecolor{ansi-blue}{HTML}{208FFB}
    \definecolor{ansi-blue-intense}{HTML}{0065CA}
    \definecolor{ansi-magenta}{HTML}{D160C4}
    \definecolor{ansi-magenta-intense}{HTML}{A03196}
    \definecolor{ansi-cyan}{HTML}{60C6C8}
    \definecolor{ansi-cyan-intense}{HTML}{258F8F}
    \definecolor{ansi-white}{HTML}{C5C1B4}
    \definecolor{ansi-white-intense}{HTML}{A1A6B2}
    \definecolor{ansi-default-inverse-fg}{HTML}{FFFFFF}
    \definecolor{ansi-default-inverse-bg}{HTML}{000000}

    % commands and environments needed by pandoc snippets
    % extracted from the output of `pandoc -s`
    \providecommand{\tightlist}{%
      \setlength{\itemsep}{0pt}\setlength{\parskip}{0pt}}
    \DefineVerbatimEnvironment{Highlighting}{Verbatim}{commandchars=\\\{\}}
    % Add ',fontsize=\small' for more characters per line
    \newenvironment{Shaded}{}{}
    \newcommand{\KeywordTok}[1]{\textcolor[rgb]{0.00,0.44,0.13}{\textbf{{#1}}}}
    \newcommand{\DataTypeTok}[1]{\textcolor[rgb]{0.56,0.13,0.00}{{#1}}}
    \newcommand{\DecValTok}[1]{\textcolor[rgb]{0.25,0.63,0.44}{{#1}}}
    \newcommand{\BaseNTok}[1]{\textcolor[rgb]{0.25,0.63,0.44}{{#1}}}
    \newcommand{\FloatTok}[1]{\textcolor[rgb]{0.25,0.63,0.44}{{#1}}}
    \newcommand{\CharTok}[1]{\textcolor[rgb]{0.25,0.44,0.63}{{#1}}}
    \newcommand{\StringTok}[1]{\textcolor[rgb]{0.25,0.44,0.63}{{#1}}}
    \newcommand{\CommentTok}[1]{\textcolor[rgb]{0.38,0.63,0.69}{\textit{{#1}}}}
    \newcommand{\OtherTok}[1]{\textcolor[rgb]{0.00,0.44,0.13}{{#1}}}
    \newcommand{\AlertTok}[1]{\textcolor[rgb]{1.00,0.00,0.00}{\textbf{{#1}}}}
    \newcommand{\FunctionTok}[1]{\textcolor[rgb]{0.02,0.16,0.49}{{#1}}}
    \newcommand{\RegionMarkerTok}[1]{{#1}}
    \newcommand{\ErrorTok}[1]{\textcolor[rgb]{1.00,0.00,0.00}{\textbf{{#1}}}}
    \newcommand{\NormalTok}[1]{{#1}}
    
    % Additional commands for more recent versions of Pandoc
    \newcommand{\ConstantTok}[1]{\textcolor[rgb]{0.53,0.00,0.00}{{#1}}}
    \newcommand{\SpecialCharTok}[1]{\textcolor[rgb]{0.25,0.44,0.63}{{#1}}}
    \newcommand{\VerbatimStringTok}[1]{\textcolor[rgb]{0.25,0.44,0.63}{{#1}}}
    \newcommand{\SpecialStringTok}[1]{\textcolor[rgb]{0.73,0.40,0.53}{{#1}}}
    \newcommand{\ImportTok}[1]{{#1}}
    \newcommand{\DocumentationTok}[1]{\textcolor[rgb]{0.73,0.13,0.13}{\textit{{#1}}}}
    \newcommand{\AnnotationTok}[1]{\textcolor[rgb]{0.38,0.63,0.69}{\textbf{\textit{{#1}}}}}
    \newcommand{\CommentVarTok}[1]{\textcolor[rgb]{0.38,0.63,0.69}{\textbf{\textit{{#1}}}}}
    \newcommand{\VariableTok}[1]{\textcolor[rgb]{0.10,0.09,0.49}{{#1}}}
    \newcommand{\ControlFlowTok}[1]{\textcolor[rgb]{0.00,0.44,0.13}{\textbf{{#1}}}}
    \newcommand{\OperatorTok}[1]{\textcolor[rgb]{0.40,0.40,0.40}{{#1}}}
    \newcommand{\BuiltInTok}[1]{{#1}}
    \newcommand{\ExtensionTok}[1]{{#1}}
    \newcommand{\PreprocessorTok}[1]{\textcolor[rgb]{0.74,0.48,0.00}{{#1}}}
    \newcommand{\AttributeTok}[1]{\textcolor[rgb]{0.49,0.56,0.16}{{#1}}}
    \newcommand{\InformationTok}[1]{\textcolor[rgb]{0.38,0.63,0.69}{\textbf{\textit{{#1}}}}}
    \newcommand{\WarningTok}[1]{\textcolor[rgb]{0.38,0.63,0.69}{\textbf{\textit{{#1}}}}}
    
    
    % Define a nice break command that doesn't care if a line doesn't already
    % exist.
    \def\br{\hspace*{\fill} \\* }
    % Math Jax compatibility definitions
    \def\gt{>}
    \def\lt{<}
    \let\Oldtex\TeX
    \let\Oldlatex\LaTeX
    \renewcommand{\TeX}{\textrm{\Oldtex}}
    \renewcommand{\LaTeX}{\textrm{\Oldlatex}}
    % Document parameters
    % Document title
    \title{HW4\_Code}
    
    
    
    
    
% Pygments definitions
\makeatletter
\def\PY@reset{\let\PY@it=\relax \let\PY@bf=\relax%
    \let\PY@ul=\relax \let\PY@tc=\relax%
    \let\PY@bc=\relax \let\PY@ff=\relax}
\def\PY@tok#1{\csname PY@tok@#1\endcsname}
\def\PY@toks#1+{\ifx\relax#1\empty\else%
    \PY@tok{#1}\expandafter\PY@toks\fi}
\def\PY@do#1{\PY@bc{\PY@tc{\PY@ul{%
    \PY@it{\PY@bf{\PY@ff{#1}}}}}}}
\def\PY#1#2{\PY@reset\PY@toks#1+\relax+\PY@do{#2}}

\expandafter\def\csname PY@tok@w\endcsname{\def\PY@tc##1{\textcolor[rgb]{0.73,0.73,0.73}{##1}}}
\expandafter\def\csname PY@tok@c\endcsname{\let\PY@it=\textit\def\PY@tc##1{\textcolor[rgb]{0.25,0.50,0.50}{##1}}}
\expandafter\def\csname PY@tok@cp\endcsname{\def\PY@tc##1{\textcolor[rgb]{0.74,0.48,0.00}{##1}}}
\expandafter\def\csname PY@tok@k\endcsname{\let\PY@bf=\textbf\def\PY@tc##1{\textcolor[rgb]{0.00,0.50,0.00}{##1}}}
\expandafter\def\csname PY@tok@kp\endcsname{\def\PY@tc##1{\textcolor[rgb]{0.00,0.50,0.00}{##1}}}
\expandafter\def\csname PY@tok@kt\endcsname{\def\PY@tc##1{\textcolor[rgb]{0.69,0.00,0.25}{##1}}}
\expandafter\def\csname PY@tok@o\endcsname{\def\PY@tc##1{\textcolor[rgb]{0.40,0.40,0.40}{##1}}}
\expandafter\def\csname PY@tok@ow\endcsname{\let\PY@bf=\textbf\def\PY@tc##1{\textcolor[rgb]{0.67,0.13,1.00}{##1}}}
\expandafter\def\csname PY@tok@nb\endcsname{\def\PY@tc##1{\textcolor[rgb]{0.00,0.50,0.00}{##1}}}
\expandafter\def\csname PY@tok@nf\endcsname{\def\PY@tc##1{\textcolor[rgb]{0.00,0.00,1.00}{##1}}}
\expandafter\def\csname PY@tok@nc\endcsname{\let\PY@bf=\textbf\def\PY@tc##1{\textcolor[rgb]{0.00,0.00,1.00}{##1}}}
\expandafter\def\csname PY@tok@nn\endcsname{\let\PY@bf=\textbf\def\PY@tc##1{\textcolor[rgb]{0.00,0.00,1.00}{##1}}}
\expandafter\def\csname PY@tok@ne\endcsname{\let\PY@bf=\textbf\def\PY@tc##1{\textcolor[rgb]{0.82,0.25,0.23}{##1}}}
\expandafter\def\csname PY@tok@nv\endcsname{\def\PY@tc##1{\textcolor[rgb]{0.10,0.09,0.49}{##1}}}
\expandafter\def\csname PY@tok@no\endcsname{\def\PY@tc##1{\textcolor[rgb]{0.53,0.00,0.00}{##1}}}
\expandafter\def\csname PY@tok@nl\endcsname{\def\PY@tc##1{\textcolor[rgb]{0.63,0.63,0.00}{##1}}}
\expandafter\def\csname PY@tok@ni\endcsname{\let\PY@bf=\textbf\def\PY@tc##1{\textcolor[rgb]{0.60,0.60,0.60}{##1}}}
\expandafter\def\csname PY@tok@na\endcsname{\def\PY@tc##1{\textcolor[rgb]{0.49,0.56,0.16}{##1}}}
\expandafter\def\csname PY@tok@nt\endcsname{\let\PY@bf=\textbf\def\PY@tc##1{\textcolor[rgb]{0.00,0.50,0.00}{##1}}}
\expandafter\def\csname PY@tok@nd\endcsname{\def\PY@tc##1{\textcolor[rgb]{0.67,0.13,1.00}{##1}}}
\expandafter\def\csname PY@tok@s\endcsname{\def\PY@tc##1{\textcolor[rgb]{0.73,0.13,0.13}{##1}}}
\expandafter\def\csname PY@tok@sd\endcsname{\let\PY@it=\textit\def\PY@tc##1{\textcolor[rgb]{0.73,0.13,0.13}{##1}}}
\expandafter\def\csname PY@tok@si\endcsname{\let\PY@bf=\textbf\def\PY@tc##1{\textcolor[rgb]{0.73,0.40,0.53}{##1}}}
\expandafter\def\csname PY@tok@se\endcsname{\let\PY@bf=\textbf\def\PY@tc##1{\textcolor[rgb]{0.73,0.40,0.13}{##1}}}
\expandafter\def\csname PY@tok@sr\endcsname{\def\PY@tc##1{\textcolor[rgb]{0.73,0.40,0.53}{##1}}}
\expandafter\def\csname PY@tok@ss\endcsname{\def\PY@tc##1{\textcolor[rgb]{0.10,0.09,0.49}{##1}}}
\expandafter\def\csname PY@tok@sx\endcsname{\def\PY@tc##1{\textcolor[rgb]{0.00,0.50,0.00}{##1}}}
\expandafter\def\csname PY@tok@m\endcsname{\def\PY@tc##1{\textcolor[rgb]{0.40,0.40,0.40}{##1}}}
\expandafter\def\csname PY@tok@gh\endcsname{\let\PY@bf=\textbf\def\PY@tc##1{\textcolor[rgb]{0.00,0.00,0.50}{##1}}}
\expandafter\def\csname PY@tok@gu\endcsname{\let\PY@bf=\textbf\def\PY@tc##1{\textcolor[rgb]{0.50,0.00,0.50}{##1}}}
\expandafter\def\csname PY@tok@gd\endcsname{\def\PY@tc##1{\textcolor[rgb]{0.63,0.00,0.00}{##1}}}
\expandafter\def\csname PY@tok@gi\endcsname{\def\PY@tc##1{\textcolor[rgb]{0.00,0.63,0.00}{##1}}}
\expandafter\def\csname PY@tok@gr\endcsname{\def\PY@tc##1{\textcolor[rgb]{1.00,0.00,0.00}{##1}}}
\expandafter\def\csname PY@tok@ge\endcsname{\let\PY@it=\textit}
\expandafter\def\csname PY@tok@gs\endcsname{\let\PY@bf=\textbf}
\expandafter\def\csname PY@tok@gp\endcsname{\let\PY@bf=\textbf\def\PY@tc##1{\textcolor[rgb]{0.00,0.00,0.50}{##1}}}
\expandafter\def\csname PY@tok@go\endcsname{\def\PY@tc##1{\textcolor[rgb]{0.53,0.53,0.53}{##1}}}
\expandafter\def\csname PY@tok@gt\endcsname{\def\PY@tc##1{\textcolor[rgb]{0.00,0.27,0.87}{##1}}}
\expandafter\def\csname PY@tok@err\endcsname{\def\PY@bc##1{\setlength{\fboxsep}{0pt}\fcolorbox[rgb]{1.00,0.00,0.00}{1,1,1}{\strut ##1}}}
\expandafter\def\csname PY@tok@kc\endcsname{\let\PY@bf=\textbf\def\PY@tc##1{\textcolor[rgb]{0.00,0.50,0.00}{##1}}}
\expandafter\def\csname PY@tok@kd\endcsname{\let\PY@bf=\textbf\def\PY@tc##1{\textcolor[rgb]{0.00,0.50,0.00}{##1}}}
\expandafter\def\csname PY@tok@kn\endcsname{\let\PY@bf=\textbf\def\PY@tc##1{\textcolor[rgb]{0.00,0.50,0.00}{##1}}}
\expandafter\def\csname PY@tok@kr\endcsname{\let\PY@bf=\textbf\def\PY@tc##1{\textcolor[rgb]{0.00,0.50,0.00}{##1}}}
\expandafter\def\csname PY@tok@bp\endcsname{\def\PY@tc##1{\textcolor[rgb]{0.00,0.50,0.00}{##1}}}
\expandafter\def\csname PY@tok@fm\endcsname{\def\PY@tc##1{\textcolor[rgb]{0.00,0.00,1.00}{##1}}}
\expandafter\def\csname PY@tok@vc\endcsname{\def\PY@tc##1{\textcolor[rgb]{0.10,0.09,0.49}{##1}}}
\expandafter\def\csname PY@tok@vg\endcsname{\def\PY@tc##1{\textcolor[rgb]{0.10,0.09,0.49}{##1}}}
\expandafter\def\csname PY@tok@vi\endcsname{\def\PY@tc##1{\textcolor[rgb]{0.10,0.09,0.49}{##1}}}
\expandafter\def\csname PY@tok@vm\endcsname{\def\PY@tc##1{\textcolor[rgb]{0.10,0.09,0.49}{##1}}}
\expandafter\def\csname PY@tok@sa\endcsname{\def\PY@tc##1{\textcolor[rgb]{0.73,0.13,0.13}{##1}}}
\expandafter\def\csname PY@tok@sb\endcsname{\def\PY@tc##1{\textcolor[rgb]{0.73,0.13,0.13}{##1}}}
\expandafter\def\csname PY@tok@sc\endcsname{\def\PY@tc##1{\textcolor[rgb]{0.73,0.13,0.13}{##1}}}
\expandafter\def\csname PY@tok@dl\endcsname{\def\PY@tc##1{\textcolor[rgb]{0.73,0.13,0.13}{##1}}}
\expandafter\def\csname PY@tok@s2\endcsname{\def\PY@tc##1{\textcolor[rgb]{0.73,0.13,0.13}{##1}}}
\expandafter\def\csname PY@tok@sh\endcsname{\def\PY@tc##1{\textcolor[rgb]{0.73,0.13,0.13}{##1}}}
\expandafter\def\csname PY@tok@s1\endcsname{\def\PY@tc##1{\textcolor[rgb]{0.73,0.13,0.13}{##1}}}
\expandafter\def\csname PY@tok@mb\endcsname{\def\PY@tc##1{\textcolor[rgb]{0.40,0.40,0.40}{##1}}}
\expandafter\def\csname PY@tok@mf\endcsname{\def\PY@tc##1{\textcolor[rgb]{0.40,0.40,0.40}{##1}}}
\expandafter\def\csname PY@tok@mh\endcsname{\def\PY@tc##1{\textcolor[rgb]{0.40,0.40,0.40}{##1}}}
\expandafter\def\csname PY@tok@mi\endcsname{\def\PY@tc##1{\textcolor[rgb]{0.40,0.40,0.40}{##1}}}
\expandafter\def\csname PY@tok@il\endcsname{\def\PY@tc##1{\textcolor[rgb]{0.40,0.40,0.40}{##1}}}
\expandafter\def\csname PY@tok@mo\endcsname{\def\PY@tc##1{\textcolor[rgb]{0.40,0.40,0.40}{##1}}}
\expandafter\def\csname PY@tok@ch\endcsname{\let\PY@it=\textit\def\PY@tc##1{\textcolor[rgb]{0.25,0.50,0.50}{##1}}}
\expandafter\def\csname PY@tok@cm\endcsname{\let\PY@it=\textit\def\PY@tc##1{\textcolor[rgb]{0.25,0.50,0.50}{##1}}}
\expandafter\def\csname PY@tok@cpf\endcsname{\let\PY@it=\textit\def\PY@tc##1{\textcolor[rgb]{0.25,0.50,0.50}{##1}}}
\expandafter\def\csname PY@tok@c1\endcsname{\let\PY@it=\textit\def\PY@tc##1{\textcolor[rgb]{0.25,0.50,0.50}{##1}}}
\expandafter\def\csname PY@tok@cs\endcsname{\let\PY@it=\textit\def\PY@tc##1{\textcolor[rgb]{0.25,0.50,0.50}{##1}}}

\def\PYZbs{\char`\\}
\def\PYZus{\char`\_}
\def\PYZob{\char`\{}
\def\PYZcb{\char`\}}
\def\PYZca{\char`\^}
\def\PYZam{\char`\&}
\def\PYZlt{\char`\<}
\def\PYZgt{\char`\>}
\def\PYZsh{\char`\#}
\def\PYZpc{\char`\%}
\def\PYZdl{\char`\$}
\def\PYZhy{\char`\-}
\def\PYZsq{\char`\'}
\def\PYZdq{\char`\"}
\def\PYZti{\char`\~}
% for compatibility with earlier versions
\def\PYZat{@}
\def\PYZlb{[}
\def\PYZrb{]}
\makeatother


    % For linebreaks inside Verbatim environment from package fancyvrb. 
    \makeatletter
        \newbox\Wrappedcontinuationbox 
        \newbox\Wrappedvisiblespacebox 
        \newcommand*\Wrappedvisiblespace {\textcolor{red}{\textvisiblespace}} 
        \newcommand*\Wrappedcontinuationsymbol {\textcolor{red}{\llap{\tiny$\m@th\hookrightarrow$}}} 
        \newcommand*\Wrappedcontinuationindent {3ex } 
        \newcommand*\Wrappedafterbreak {\kern\Wrappedcontinuationindent\copy\Wrappedcontinuationbox} 
        % Take advantage of the already applied Pygments mark-up to insert 
        % potential linebreaks for TeX processing. 
        %        {, <, #, %, $, ' and ": go to next line. 
        %        _, }, ^, &, >, - and ~: stay at end of broken line. 
        % Use of \textquotesingle for straight quote. 
        \newcommand*\Wrappedbreaksatspecials {% 
            \def\PYGZus{\discretionary{\char`\_}{\Wrappedafterbreak}{\char`\_}}% 
            \def\PYGZob{\discretionary{}{\Wrappedafterbreak\char`\{}{\char`\{}}% 
            \def\PYGZcb{\discretionary{\char`\}}{\Wrappedafterbreak}{\char`\}}}% 
            \def\PYGZca{\discretionary{\char`\^}{\Wrappedafterbreak}{\char`\^}}% 
            \def\PYGZam{\discretionary{\char`\&}{\Wrappedafterbreak}{\char`\&}}% 
            \def\PYGZlt{\discretionary{}{\Wrappedafterbreak\char`\<}{\char`\<}}% 
            \def\PYGZgt{\discretionary{\char`\>}{\Wrappedafterbreak}{\char`\>}}% 
            \def\PYGZsh{\discretionary{}{\Wrappedafterbreak\char`\#}{\char`\#}}% 
            \def\PYGZpc{\discretionary{}{\Wrappedafterbreak\char`\%}{\char`\%}}% 
            \def\PYGZdl{\discretionary{}{\Wrappedafterbreak\char`\$}{\char`\$}}% 
            \def\PYGZhy{\discretionary{\char`\-}{\Wrappedafterbreak}{\char`\-}}% 
            \def\PYGZsq{\discretionary{}{\Wrappedafterbreak\textquotesingle}{\textquotesingle}}% 
            \def\PYGZdq{\discretionary{}{\Wrappedafterbreak\char`\"}{\char`\"}}% 
            \def\PYGZti{\discretionary{\char`\~}{\Wrappedafterbreak}{\char`\~}}% 
        } 
        % Some characters . , ; ? ! / are not pygmentized. 
        % This macro makes them "active" and they will insert potential linebreaks 
        \newcommand*\Wrappedbreaksatpunct {% 
            \lccode`\~`\.\lowercase{\def~}{\discretionary{\hbox{\char`\.}}{\Wrappedafterbreak}{\hbox{\char`\.}}}% 
            \lccode`\~`\,\lowercase{\def~}{\discretionary{\hbox{\char`\,}}{\Wrappedafterbreak}{\hbox{\char`\,}}}% 
            \lccode`\~`\;\lowercase{\def~}{\discretionary{\hbox{\char`\;}}{\Wrappedafterbreak}{\hbox{\char`\;}}}% 
            \lccode`\~`\:\lowercase{\def~}{\discretionary{\hbox{\char`\:}}{\Wrappedafterbreak}{\hbox{\char`\:}}}% 
            \lccode`\~`\?\lowercase{\def~}{\discretionary{\hbox{\char`\?}}{\Wrappedafterbreak}{\hbox{\char`\?}}}% 
            \lccode`\~`\!\lowercase{\def~}{\discretionary{\hbox{\char`\!}}{\Wrappedafterbreak}{\hbox{\char`\!}}}% 
            \lccode`\~`\/\lowercase{\def~}{\discretionary{\hbox{\char`\/}}{\Wrappedafterbreak}{\hbox{\char`\/}}}% 
            \catcode`\.\active
            \catcode`\,\active 
            \catcode`\;\active
            \catcode`\:\active
            \catcode`\?\active
            \catcode`\!\active
            \catcode`\/\active 
            \lccode`\~`\~ 	
        }
    \makeatother

    \let\OriginalVerbatim=\Verbatim
    \makeatletter
    \renewcommand{\Verbatim}[1][1]{%
        %\parskip\z@skip
        \sbox\Wrappedcontinuationbox {\Wrappedcontinuationsymbol}%
        \sbox\Wrappedvisiblespacebox {\FV@SetupFont\Wrappedvisiblespace}%
        \def\FancyVerbFormatLine ##1{\hsize\linewidth
            \vtop{\raggedright\hyphenpenalty\z@\exhyphenpenalty\z@
                \doublehyphendemerits\z@\finalhyphendemerits\z@
                \strut ##1\strut}%
        }%
        % If the linebreak is at a space, the latter will be displayed as visible
        % space at end of first line, and a continuation symbol starts next line.
        % Stretch/shrink are however usually zero for typewriter font.
        \def\FV@Space {%
            \nobreak\hskip\z@ plus\fontdimen3\font minus\fontdimen4\font
            \discretionary{\copy\Wrappedvisiblespacebox}{\Wrappedafterbreak}
            {\kern\fontdimen2\font}%
        }%
        
        % Allow breaks at special characters using \PYG... macros.
        \Wrappedbreaksatspecials
        % Breaks at punctuation characters . , ; ? ! and / need catcode=\active 	
        \OriginalVerbatim[#1,codes*=\Wrappedbreaksatpunct]%
    }
    \makeatother

    % Exact colors from NB
    \definecolor{incolor}{HTML}{303F9F}
    \definecolor{outcolor}{HTML}{D84315}
    \definecolor{cellborder}{HTML}{CFCFCF}
    \definecolor{cellbackground}{HTML}{F7F7F7}
    
    % prompt
    \makeatletter
    \newcommand{\boxspacing}{\kern\kvtcb@left@rule\kern\kvtcb@boxsep}
    \makeatother
    \newcommand{\prompt}[4]{
        \ttfamily\llap{{\color{#2}[#3]:\hspace{3pt}#4}}\vspace{-\baselineskip}
    }
    

    
    % Prevent overflowing lines due to hard-to-break entities
    \sloppy 
    % Setup hyperref package
    \hypersetup{
      breaklinks=true,  % so long urls are correctly broken across lines
      colorlinks=true,
      urlcolor=urlcolor,
      linkcolor=linkcolor,
      citecolor=citecolor,
      }
    % Slightly bigger margins than the latex defaults
    
    \geometry{verbose,tmargin=1in,bmargin=1in,lmargin=1in,rmargin=1in}
    
        %%%%%%%%%%%%%%%%%%%%%%%%%%%%%%%%%%%%%%
    \newcommand{\hmwkTitle}{HW\ \#4}
    \newcommand{\hmwkDueDate}{}
    \newcommand{\hmwkClass}{CpE 520}
    \newcommand{\institute}{West Virginia University}
    \newcommand{\hmwkClassInstructor}{Professor Nasser Nasrabadi}
    \newcommand{\hmwkAuthorName}{\textbf{Ali Zafari}}
    
    \title{
    	\vspace{2in}
    	\textmd{\textbf{\hmwkClass:\ \hmwkTitle}}\\
    	\vspace{0.1in}\large{\institute}\\
%    	\vspace{0.1in}\large{\textit{\hmwkClassInstructor}}
    	\vspace{3in}
    }
    
    \author{\hmwkAuthorName}
    \date{}
    %%%%%%%%%%%%%%%%%%%%%%%%%%%%%%%%%%%%%%

\begin{document}
	
    %\maketitle
    \pagenumbering{gobble}% prevent cover page of numbering
    \maketitle
    \pagebreak % let cover page free to the end of page
    \pagenumbering{arabic} % start page numbering again from 1 and print them!
    \tableofcontents
    \pagebreak
    
    

    
    \hypertarget{mnist-dataset}{%
\section{\texorpdfstring{\emph{MNIST}
Dataset}{MNIST Dataset}}\label{mnist-dataset}}

    \hypertarget{installing-mnsit-package}{%
\subsection{\texorpdfstring{Installing \emph{MNSIT}
Package}{Installing MNSIT Package}}\label{installing-mnsit-package}}

At first to load the MNIST dataset, we install the mnist package of
python:

    \begin{tcolorbox}[breakable, size=fbox, boxrule=1pt, pad at break*=1mm,colback=cellbackground, colframe=cellborder]
\prompt{In}{incolor}{1}{\boxspacing}
\begin{Verbatim}[commandchars=\\\{\}]
\PY{o}{!}pip install mnist
\end{Verbatim}
\end{tcolorbox}

    \begin{Verbatim}[commandchars=\\\{\}]
Collecting mnist
  Downloading https://files.pythonhosted.org/packages/c6/c4/5db3bfe009f8d71f1d53
2bbadbd0ec203764bba3a469e4703a889db8e5e0/mnist-0.2.2-py2.py3-none-any.whl
Requirement already satisfied: numpy in /usr/local/lib/python3.6/dist-packages
(from mnist) (1.18.2)
Installing collected packages: mnist
Successfully installed mnist-0.2.2
    \end{Verbatim}

    \hypertarget{loading-the-mnist-dataset}{%
\subsection{\texorpdfstring{Loading the \emph{MNIST}
dataset}{Loading the MNIST dataset}}\label{loading-the-mnist-dataset}}

    The code below will import the necessary libraries and loads the
training part of the MNIST dataset which will be in a matrix shape of
(60000, 28, 28)

    \begin{tcolorbox}[breakable, size=fbox, boxrule=1pt, pad at break*=1mm,colback=cellbackground, colframe=cellborder]
\prompt{In}{incolor}{ }{\boxspacing}
\begin{Verbatim}[commandchars=\\\{\}]
\PY{k+kn}{import} \PY{n+nn}{mnist}
\PY{k+kn}{import} \PY{n+nn}{os}
\PY{k+kn}{import} \PY{n+nn}{numpy} \PY{k}{as} \PY{n+nn}{np}
\PY{k+kn}{import} \PY{n+nn}{matplotlib}\PY{n+nn}{.}\PY{n+nn}{pyplot} \PY{k}{as} \PY{n+nn}{plt}

\PY{n}{mnist}\PY{o}{.}\PY{n}{temporary\PYZus{}dir} \PY{o}{=} \PY{k}{lambda}\PY{p}{:} \PY{n}{os}\PY{o}{.}\PY{n}{getcwd}\PY{p}{(}\PY{p}{)}

\PY{c+c1}{\PYZsh{} Each of these functions first downloads the data and returns a numpy array.}
\PY{n}{train\PYZus{}images} \PY{o}{=} \PY{n}{mnist}\PY{o}{.}\PY{n}{train\PYZus{}images}\PY{p}{(}\PY{p}{)}

\PY{k}{assert} \PY{n}{train\PYZus{}images}\PY{o}{.}\PY{n}{shape} \PY{o}{==} \PY{p}{(}\PY{l+m+mi}{60000}\PY{p}{,} \PY{l+m+mi}{28}\PY{p}{,} \PY{l+m+mi}{28}\PY{p}{)}\PY{p}{,} \PY{l+s+s1}{\PYZsq{}}\PY{l+s+s1}{train\PYZus{}imgages shape is not correct}\PY{l+s+s1}{\PYZsq{}}
\end{Verbatim}
\end{tcolorbox}

    \hypertarget{showing-a-sample-of-dataset}{%
\subsection{Showing a Sample of
Dataset}\label{showing-a-sample-of-dataset}}

    This code snippet will prints the first 9 images as an example:

    \begin{tcolorbox}[breakable, size=fbox, boxrule=1pt, pad at break*=1mm,colback=cellbackground, colframe=cellborder]
\prompt{In}{incolor}{3}{\boxspacing}
\begin{Verbatim}[commandchars=\\\{\}]
\PY{n}{fig}\PY{p}{,} \PY{n}{ax} \PY{o}{=} \PY{n}{plt}\PY{o}{.}\PY{n}{subplots}\PY{p}{(}\PY{l+m+mi}{3}\PY{p}{,} \PY{l+m+mi}{3}\PY{p}{)}
\PY{k}{for} \PY{n}{i} \PY{o+ow}{in} \PY{n+nb}{range}\PY{p}{(}\PY{l+m+mi}{3}\PY{p}{)}\PY{p}{:}
    \PY{k}{for} \PY{n}{j} \PY{o+ow}{in} \PY{n+nb}{range}\PY{p}{(}\PY{l+m+mi}{3}\PY{p}{)}\PY{p}{:}
        \PY{n}{ax}\PY{p}{[}\PY{n}{i}\PY{p}{]}\PY{p}{[}\PY{n}{j}\PY{p}{]}\PY{o}{.}\PY{n}{imshow}\PY{p}{(}\PY{n}{train\PYZus{}images}\PY{p}{[}\PY{l+m+mi}{3} \PY{o}{*} \PY{n}{i} \PY{o}{+} \PY{n}{j}\PY{p}{]}\PY{p}{,} \PY{n}{cmap}\PY{o}{=}\PY{l+s+s1}{\PYZsq{}}\PY{l+s+s1}{Greys}\PY{l+s+s1}{\PYZsq{}}\PY{p}{)}
        \PY{n}{ax}\PY{p}{[}\PY{n}{i}\PY{p}{]}\PY{p}{[}\PY{n}{j}\PY{p}{]}\PY{o}{.}\PY{n}{xaxis}\PY{o}{.}\PY{n}{set\PYZus{}visible}\PY{p}{(}\PY{k+kc}{False}\PY{p}{)}
        \PY{n}{ax}\PY{p}{[}\PY{n}{i}\PY{p}{]}\PY{p}{[}\PY{n}{j}\PY{p}{]}\PY{o}{.}\PY{n}{yaxis}\PY{o}{.}\PY{n}{set\PYZus{}visible}\PY{p}{(}\PY{k+kc}{False}\PY{p}{)}
\end{Verbatim}
\end{tcolorbox}

    \begin{center}
    \adjustimage{max size={0.9\linewidth}{0.9\paperheight}}{output_8_0.png}
    \end{center}
    { \hspace*{\fill} \\}
    
    \hypertarget{reshaping-the-training-images}{%
\subsection{Reshaping the Training
Images}\label{reshaping-the-training-images}}

    To do matrix operations on this data we reshape each image of 28x28 to a
vector of 784 element, so the new shape of the data will be like (60000,
784).

    \begin{tcolorbox}[breakable, size=fbox, boxrule=1pt, pad at break*=1mm,colback=cellbackground, colframe=cellborder]
\prompt{In}{incolor}{4}{\boxspacing}
\begin{Verbatim}[commandchars=\\\{\}]
\PY{n}{train\PYZus{}images\PYZus{}reshaped} \PY{o}{=} \PY{n}{train\PYZus{}images}\PY{o}{.}\PY{n}{reshape}\PY{p}{(}\PY{p}{(}\PY{n}{train\PYZus{}images}\PY{o}{.}\PY{n}{shape}\PY{p}{[}\PY{l+m+mi}{0}\PY{p}{]}\PY{p}{,} \PY{l+m+mi}{28}\PY{o}{*}\PY{l+m+mi}{28}\PY{p}{)}\PY{p}{)}
\PY{n+nb}{print}\PY{p}{(}\PY{l+s+sa}{f}\PY{l+s+s1}{\PYZsq{}}\PY{l+s+s1}{Train images before reshape: }\PY{l+s+si}{\PYZob{}train\PYZus{}images.shape\PYZcb{}}\PY{l+s+s1}{\PYZsq{}}\PY{p}{)}
\PY{n+nb}{print}\PY{p}{(}\PY{l+s+sa}{f}\PY{l+s+s1}{\PYZsq{}}\PY{l+s+s1}{Train images after  reshape: }\PY{l+s+si}{\PYZob{}train\PYZus{}images\PYZus{}reshaped.shape\PYZcb{}}\PY{l+s+s1}{\PYZsq{}}\PY{p}{)}

\PY{k}{assert} \PY{n+nb}{all}\PY{p}{(}\PY{n}{train\PYZus{}images}\PY{p}{[}\PY{l+m+mi}{0}\PY{p}{,} \PY{l+m+mi}{20}\PY{p}{]} \PY{o}{==} \PY{n}{train\PYZus{}images\PYZus{}reshaped}\PY{p}{[}\PY{l+m+mi}{0}\PY{p}{,} \PY{p}{(}\PY{l+m+mi}{20}\PY{p}{)}\PY{o}{*}\PY{l+m+mi}{28}\PY{p}{:}\PY{p}{(}\PY{l+m+mi}{20}\PY{o}{+}\PY{l+m+mi}{1}\PY{p}{)}\PY{o}{*}\PY{l+m+mi}{28}\PY{p}{]}\PY{p}{)}\PY{p}{,} \PY{l+s+s2}{\PYZdq{}}\PY{l+s+s2}{train reshaping is wrong}\PY{l+s+s2}{\PYZdq{}}
\end{Verbatim}
\end{tcolorbox}

    \begin{Verbatim}[commandchars=\\\{\}]
Train images before reshape: (60000, 28, 28)
Train images after  reshape: (60000, 784)
    \end{Verbatim}
\pagebreak
    \hypertarget{implementing-k-means-algorithm-in-functions}{%
\section{Implementing K-means Algorithm in
Functions}\label{implementing-k-means-algorithm-in-functions}}

    All the functions that implement the K-means algorithm are listed below,
it is sufficient to call the function
\textbf{\emph{kmeans\_algorithm\_with\_distortion}} and it will use the
other functions. The output of this function is the final centroids and
final cluster number that is assigned to each image.

    \begin{tcolorbox}[breakable, size=fbox, boxrule=1pt, pad at break*=1mm,colback=cellbackground, colframe=cellborder]
\prompt{In}{incolor}{ }{\boxspacing}
\begin{Verbatim}[commandchars=\\\{\}]
\PY{k}{def} \PY{n+nf}{initialize\PYZus{}centroids}\PY{p}{(}\PY{n}{X}\PY{p}{,} \PY{n}{K}\PY{p}{)}\PY{p}{:}
    \PY{k}{return} \PY{n}{X}\PY{p}{[}\PY{n}{np}\PY{o}{.}\PY{n}{random}\PY{o}{.}\PY{n}{choice}\PY{p}{(}\PY{n}{X}\PY{o}{.}\PY{n}{shape}\PY{p}{[}\PY{l+m+mi}{0}\PY{p}{]}\PY{p}{,} \PY{n}{K}\PY{p}{,} \PY{n}{replace}\PY{o}{=}\PY{k+kc}{False}\PY{p}{)}\PY{p}{,} \PY{p}{:}\PY{p}{]}


\PY{k}{def} \PY{n+nf}{find\PYZus{}closest\PYZus{}centroids}\PY{p}{(}\PY{n}{X}\PY{p}{,} \PY{n}{centroids}\PY{p}{)}\PY{p}{:}
    \PY{c+c1}{\PYZsh{}print(\PYZsq{}find\PYZus{}closest\PYZus{}centroids\PYZus{}started\PYZsq{})}
    \PY{n}{K} \PY{o}{=} \PY{n}{centroids}\PY{o}{.}\PY{n}{shape}\PY{p}{[}\PY{l+m+mi}{0}\PY{p}{]}
    \PY{n}{idx} \PY{o}{=} \PY{n}{np}\PY{o}{.}\PY{n}{zeros}\PY{p}{(}\PY{n}{X}\PY{o}{.}\PY{n}{shape}\PY{p}{[}\PY{l+m+mi}{0}\PY{p}{]}\PY{p}{)}
    
    \PY{k}{for} \PY{n}{n} \PY{o+ow}{in} \PY{n+nb}{range}\PY{p}{(}\PY{n}{X}\PY{o}{.}\PY{n}{shape}\PY{p}{[}\PY{l+m+mi}{0}\PY{p}{]}\PY{p}{)}\PY{p}{:}
        \PY{n}{min\PYZus{}distance} \PY{o}{=} \PY{n}{np}\PY{o}{.}\PY{n}{inf}
        \PY{n}{j} \PY{o}{=} \PY{l+m+mi}{0}
        \PY{k}{for} \PY{n}{m} \PY{o+ow}{in} \PY{n+nb}{range}\PY{p}{(}\PY{n}{K}\PY{p}{)}\PY{p}{:}
            
            \PY{n}{norm} \PY{o}{=} \PY{n}{np}\PY{o}{.}\PY{n}{linalg}\PY{o}{.}\PY{n}{norm}\PY{p}{(}\PY{n}{centroids}\PY{p}{[}\PY{n}{m}\PY{p}{,} \PY{p}{:}\PY{p}{]}\PY{o}{\PYZhy{}}\PY{n}{X}\PY{p}{[}\PY{n}{n}\PY{p}{,} \PY{p}{:}\PY{p}{]}\PY{p}{)}
            \PY{c+c1}{\PYZsh{}print(m, n, norm)}
            \PY{k}{if} \PY{n}{norm} \PY{o}{\PYZlt{}} \PY{n}{min\PYZus{}distance}\PY{p}{:}
                \PY{n}{min\PYZus{}distance} \PY{o}{=} \PY{n}{norm}
                \PY{n}{j} \PY{o}{=} \PY{n}{m}
        \PY{n}{idx}\PY{p}{[}\PY{n}{n}\PY{p}{]} \PY{o}{=} \PY{n}{j}
    \PY{c+c1}{\PYZsh{}print(\PYZsq{}find\PYZus{}closest\PYZus{}centroids\PYZus{}finished\PYZsq{})}
    \PY{k}{return} \PY{n}{idx}


\PY{k}{def} \PY{n+nf}{find\PYZus{}closest\PYZus{}centroids\PYZus{}numpyed}\PY{p}{(}\PY{n}{X}\PY{p}{,} \PY{n}{centroids}\PY{p}{)}\PY{p}{:} \PY{c+c1}{\PYZsh{}maybe faster! but is not memory friendly}
    \PY{c+c1}{\PYZsh{}print(\PYZsq{}find\PYZus{}closest\PYZus{}centroids\PYZus{}numpyed\PYZus{}started\PYZsq{})}
    \PY{n}{K} \PY{o}{=} \PY{n}{centroids}\PY{o}{.}\PY{n}{shape}\PY{p}{[}\PY{l+m+mi}{0}\PY{p}{]}

    \PY{n}{X\PYZus{}minus\PYZus{}Cs} \PY{o}{=} \PY{n}{np}\PY{o}{.}\PY{n}{array}\PY{p}{(}\PY{p}{[}\PY{n}{X}\PY{o}{\PYZhy{}}\PY{n}{centroids}\PY{p}{[}\PY{n}{i}\PY{p}{,} \PY{p}{:}\PY{p}{]} \PY{k}{for} \PY{n}{i} \PY{o+ow}{in} \PY{n+nb}{range}\PY{p}{(}\PY{n}{K}\PY{p}{)}\PY{p}{]}\PY{p}{)}
    \PY{n}{norms} \PY{o}{=} \PY{n}{np}\PY{o}{.}\PY{n}{linalg}\PY{o}{.}\PY{n}{norm}\PY{p}{(}\PY{n}{X\PYZus{}minus\PYZus{}Cs}\PY{p}{,} \PY{n}{axis}\PY{o}{=}\PY{l+m+mi}{2}\PY{p}{)}\PY{o}{.}\PY{n}{T}
    \PY{n}{idx} \PY{o}{=} \PY{n}{np}\PY{o}{.}\PY{n}{argmin}\PY{p}{(}\PY{n}{norms}\PY{p}{,} \PY{n}{axis}\PY{o}{=}\PY{l+m+mi}{1}\PY{p}{)}
    \PY{c+c1}{\PYZsh{}print(\PYZsq{}find\PYZus{}closest\PYZus{}centroids\PYZus{}numpyed\PYZus{}finished\PYZsq{})}
    \PY{k}{return} \PY{n}{idx}



\PY{k}{def} \PY{n+nf}{compute\PYZus{}centroids}\PY{p}{(}\PY{n}{X}\PY{p}{,} \PY{n}{idx}\PY{p}{,} \PY{n}{K}\PY{p}{,} \PY{n}{prev\PYZus{}centroids}\PY{p}{)}\PY{p}{:}
    \PY{n}{centroids} \PY{o}{=} \PY{n}{np}\PY{o}{.}\PY{n}{copy}\PY{p}{(}\PY{n}{prev\PYZus{}centroids}\PY{p}{)}
    
    \PY{k}{for} \PY{n}{i} \PY{o+ow}{in} \PY{n}{np}\PY{o}{.}\PY{n}{unique}\PY{p}{(}\PY{n}{idx}\PY{p}{)}\PY{o}{.}\PY{n}{astype}\PY{p}{(}\PY{n+nb}{int}\PY{p}{)}\PY{p}{:}
        \PY{n}{centroids}\PY{p}{[}\PY{n}{i}\PY{p}{,} \PY{p}{:}\PY{p}{]} \PY{o}{=} \PY{n}{np}\PY{o}{.}\PY{n}{mean}\PY{p}{(}\PY{n}{X}\PY{p}{[}\PY{n}{idx}\PY{o}{.}\PY{n}{astype}\PY{p}{(}\PY{n+nb}{int}\PY{p}{)} \PY{o}{==} \PY{n}{i}\PY{p}{,} \PY{p}{:}\PY{p}{]}\PY{p}{,} \PY{n}{axis}\PY{o}{=}\PY{l+m+mi}{0}\PY{p}{)}
    \PY{k}{return} \PY{n}{centroids}


\PY{k}{def} \PY{n+nf}{show\PYZus{}centroids}\PY{p}{(}\PY{n}{centroids}\PY{p}{,} \PY{n}{K}\PY{p}{)}\PY{p}{:}
    \PY{n}{fig}\PY{p}{,} \PY{n}{ax} \PY{o}{=} \PY{n}{plt}\PY{o}{.}\PY{n}{subplots}\PY{p}{(}\PY{n}{K}\PY{o}{/}\PY{o}{/}\PY{l+m+mi}{5}\PY{p}{,} \PY{l+m+mi}{5}\PY{p}{)}
    \PY{n}{ax} \PY{o}{=} \PY{n}{ax}\PY{o}{.}\PY{n}{reshape}\PY{p}{(}\PY{n}{K}\PY{o}{/}\PY{o}{/}\PY{l+m+mi}{5}\PY{p}{,} \PY{l+m+mi}{5}\PY{p}{)}
    \PY{k}{for} \PY{n}{i} \PY{o+ow}{in} \PY{n+nb}{range}\PY{p}{(}\PY{n}{K}\PY{o}{/}\PY{o}{/}\PY{l+m+mi}{5}\PY{p}{)}\PY{p}{:}
        \PY{k}{for} \PY{n}{j} \PY{o+ow}{in} \PY{n+nb}{range}\PY{p}{(}\PY{l+m+mi}{5}\PY{p}{)}\PY{p}{:}
            \PY{n}{ax}\PY{p}{[}\PY{n}{i}\PY{p}{]}\PY{p}{[}\PY{n}{j}\PY{p}{]}\PY{o}{.}\PY{n}{imshow}\PY{p}{(}\PY{n}{centroids}\PY{p}{[}\PY{l+m+mi}{5} \PY{o}{*} \PY{n}{i} \PY{o}{+} \PY{n}{j}\PY{p}{]}\PY{o}{.}\PY{n}{reshape}\PY{p}{(}\PY{l+m+mi}{28}\PY{p}{,} \PY{l+m+mi}{28}\PY{p}{)}\PY{p}{,} \PY{n}{cmap}\PY{o}{=}\PY{l+s+s1}{\PYZsq{}}\PY{l+s+s1}{Greys}\PY{l+s+s1}{\PYZsq{}}\PY{p}{)}
            \PY{n}{ax}\PY{p}{[}\PY{n}{i}\PY{p}{]}\PY{p}{[}\PY{n}{j}\PY{p}{]}\PY{o}{.}\PY{n}{xaxis}\PY{o}{.}\PY{n}{set\PYZus{}visible}\PY{p}{(}\PY{k+kc}{False}\PY{p}{)}
            \PY{n}{ax}\PY{p}{[}\PY{n}{i}\PY{p}{]}\PY{p}{[}\PY{n}{j}\PY{p}{]}\PY{o}{.}\PY{n}{yaxis}\PY{o}{.}\PY{n}{set\PYZus{}visible}\PY{p}{(}\PY{k+kc}{False}\PY{p}{)}
    \PY{n}{plt}\PY{o}{.}\PY{n}{show}\PY{p}{(}\PY{p}{)}
   

\PY{k}{def} \PY{n+nf}{distortion}\PY{p}{(}\PY{n}{final\PYZus{}centroids}\PY{p}{,} \PY{n}{final\PYZus{}idx}\PY{p}{)}\PY{p}{:}
    \PY{n}{J} \PY{o}{=} \PY{l+m+mi}{0}
    
    \PY{k}{for} \PY{n}{i} \PY{o+ow}{in} \PY{n}{np}\PY{o}{.}\PY{n}{unique}\PY{p}{(}\PY{n}{final\PYZus{}idx}\PY{p}{)}\PY{o}{.}\PY{n}{astype}\PY{p}{(}\PY{n+nb}{int}\PY{p}{)}\PY{p}{:}
        \PY{n}{X\PYZus{}desired} \PY{o}{=} \PY{n}{X}\PY{p}{[}\PY{n}{final\PYZus{}idx}\PY{o}{.}\PY{n}{astype}\PY{p}{(}\PY{n+nb}{int}\PY{p}{)} \PY{o}{==} \PY{n}{i}\PY{p}{,} \PY{p}{:}\PY{p}{]}
        \PY{n}{X\PYZus{}desired} \PY{o}{=} \PY{n}{X\PYZus{}desired}\PY{o}{\PYZhy{}}\PY{n}{final\PYZus{}centroids}\PY{p}{[}\PY{n}{i}\PY{p}{,} \PY{p}{:}\PY{p}{]}
        \PY{n}{J} \PY{o}{+}\PY{o}{=} \PY{n}{np}\PY{o}{.}\PY{n}{mean}\PY{p}{(}\PY{n}{X\PYZus{}desired}\PY{o}{*}\PY{o}{*}\PY{l+m+mi}{2}\PY{p}{)}
    \PY{k}{return} \PY{n}{J}


\PY{k}{def} \PY{n+nf}{kmeans\PYZus{}algorithm}\PY{p}{(}\PY{n}{X}\PY{p}{,} \PY{n}{initial\PYZus{}centroids}\PY{p}{,} \PY{n}{iner\PYZus{}max\PYZus{}iters}\PY{p}{,} \PY{n}{K}\PY{p}{)}\PY{p}{:}
    \PY{n}{prev\PYZus{}centroids} \PY{o}{=} \PY{n}{initial\PYZus{}centroids}
    
    \PY{k}{for} \PY{n}{i} \PY{o+ow}{in} \PY{n+nb}{range}\PY{p}{(}\PY{n}{iner\PYZus{}max\PYZus{}iters}\PY{p}{)}\PY{p}{:}
        \PY{n}{idx} \PY{o}{=} \PY{n}{find\PYZus{}closest\PYZus{}centroids}\PY{p}{(}\PY{n}{X}\PY{p}{,} \PY{n}{prev\PYZus{}centroids}\PY{p}{)}
        \PY{n}{new\PYZus{}centroids} \PY{o}{=} \PY{n}{compute\PYZus{}centroids}\PY{p}{(}\PY{n}{X}\PY{p}{,} \PY{n}{idx}\PY{p}{,} \PY{n}{K}\PY{p}{,} \PY{n}{prev\PYZus{}centroids}\PY{p}{)}
        
        \PY{n}{centroids\PYZus{}change\PYZus{}metric} \PY{o}{=} \PY{n}{np}\PY{o}{.}\PY{n}{linalg}\PY{o}{.}\PY{n}{norm}\PY{p}{(}\PY{n}{new\PYZus{}centroids} \PY{o}{\PYZhy{}} \PY{n}{prev\PYZus{}centroids}\PY{p}{)}
        \PY{n+nb}{print}\PY{p}{(}\PY{l+s+sa}{f}\PY{l+s+s1}{\PYZsq{}}\PY{l+s+s1}{\PYZob{}}\PY{l+s+s1}{i+1\PYZcb{}th iteration: centroids got changed as }\PY{l+s+si}{\PYZob{}centroids\PYZus{}change\PYZus{}metric\PYZcb{}}\PY{l+s+s1}{.}\PY{l+s+s1}{\PYZsq{}}\PY{p}{)}
        \PY{n+nb}{print}\PY{p}{(}\PY{l+s+sa}{f}\PY{l+s+s1}{\PYZsq{}}\PY{l+s+s1}{inner J is: }\PY{l+s+s1}{\PYZob{}}\PY{l+s+s1}{distortion(new\PYZus{}centroids, idx)\PYZcb{}}\PY{l+s+s1}{\PYZsq{}}\PY{p}{)}
        \PY{k}{if} \PY{n}{centroids\PYZus{}change\PYZus{}metric} \PY{o}{\PYZlt{}} \PY{l+m+mf}{0.01}\PY{p}{:}
            \PY{n+nb}{print}\PY{p}{(}\PY{l+s+sa}{f}\PY{l+s+s1}{\PYZsq{}}\PY{l+s+s1}{K\PYZhy{}Means algorithm converged on }\PY{l+s+si}{\PYZob{}i\PYZcb{}}\PY{l+s+s1}{(th) iteration.}\PY{l+s+s1}{\PYZsq{}}\PY{p}{)}
            \PY{k}{break}
        \PY{n}{prev\PYZus{}centroids} \PY{o}{=} \PY{n}{new\PYZus{}centroids}
        \PY{n+nb}{print}\PY{p}{(}\PY{l+s+s1}{\PYZsq{}}\PY{l+s+s1}{\PYZhy{}\PYZhy{}\PYZhy{}\PYZhy{}\PYZhy{}\PYZhy{}\PYZhy{}\PYZhy{}\PYZhy{}\PYZhy{}\PYZhy{}\PYZhy{}\PYZhy{}\PYZhy{}\PYZhy{}\PYZhy{}\PYZhy{}\PYZhy{}\PYZhy{}\PYZhy{}\PYZhy{}\PYZhy{}\PYZhy{}\PYZhy{}\PYZhy{}\PYZhy{}\PYZhy{}\PYZhy{}\PYZhy{}\PYZhy{}\PYZhy{}\PYZhy{}\PYZhy{}\PYZhy{}\PYZhy{}\PYZhy{}\PYZhy{}\PYZhy{}\PYZhy{}\PYZhy{}\PYZhy{}\PYZhy{}}\PY{l+s+s1}{\PYZsq{}}\PY{p}{)}
    \PY{k}{return} \PY{n}{new\PYZus{}centroids}\PY{p}{,} \PY{n}{idx}
    
    
\PY{k}{def} \PY{n+nf}{kmeans\PYZus{}algorithm\PYZus{}with\PYZus{}distortion}\PY{p}{(}\PY{n}{X}\PY{p}{,} \PY{n}{max\PYZus{}iters}\PY{p}{,} \PY{n}{iner\PYZus{}max\PYZus{}iters}\PY{p}{,} \PY{n}{K}\PY{p}{)}\PY{p}{:}
    \PY{n}{lowest\PYZus{}J} \PY{o}{=} \PY{n}{np}\PY{o}{.}\PY{n}{inf}
    \PY{n}{new\PYZus{}centroids} \PY{o}{=} \PY{n}{np}\PY{o}{.}\PY{n}{zeros}\PY{p}{(}\PY{p}{(}\PY{n}{K}\PY{p}{,} \PY{n}{X}\PY{o}{.}\PY{n}{shape}\PY{p}{[}\PY{l+m+mi}{1}\PY{p}{]}\PY{p}{)}\PY{p}{)}
    \PY{n}{new\PYZus{}idx} \PY{o}{=} \PY{n}{np}\PY{o}{.}\PY{n}{zeros}\PY{p}{(}\PY{n}{X}\PY{o}{.}\PY{n}{shape}\PY{p}{[}\PY{l+m+mi}{0}\PY{p}{]}\PY{p}{)}
    
    \PY{k}{for} \PY{n}{i} \PY{o+ow}{in} \PY{n+nb}{range}\PY{p}{(}\PY{n}{max\PYZus{}iters}\PY{p}{)}\PY{p}{:}
        \PY{n+nb}{print}\PY{p}{(}\PY{l+s+sa}{f}\PY{l+s+s1}{\PYZsq{}}\PY{l+s+s1}{iteration }\PY{l+s+s1}{\PYZob{}}\PY{l+s+s1}{i+1\PYZcb{} of }\PY{l+s+si}{\PYZob{}max\PYZus{}iters\PYZcb{}}\PY{l+s+s1}{ is going on...}\PY{l+s+s1}{\PYZsq{}}\PY{p}{)}
        \PY{n}{new\PYZus{}initial\PYZus{}centroids} \PY{o}{=} \PY{n}{initialize\PYZus{}centroids}\PY{p}{(}\PY{n}{X}\PY{p}{,} \PY{n}{K}\PY{p}{)}
        \PY{n}{new\PYZus{}centroids}\PY{p}{,} \PY{n}{new\PYZus{}idx} \PY{o}{=} \PY{n}{kmeans\PYZus{}algorithm}\PY{p}{(}\PY{n}{X}\PY{p}{,} \PY{n}{new\PYZus{}initial\PYZus{}centroids}\PY{p}{,} \PY{n}{iner\PYZus{}max\PYZus{}iters}\PY{p}{,} \PY{n}{K}\PY{p}{)}
        \PY{n}{new\PYZus{}J} \PY{o}{=} \PY{n}{distortion}\PY{p}{(}\PY{n}{new\PYZus{}centroids}\PY{p}{,} \PY{n}{new\PYZus{}idx}\PY{p}{)}
        \PY{n+nb}{print}\PY{p}{(}\PY{l+s+sa}{f}\PY{l+s+s1}{\PYZsq{}}\PY{l+s+s1}{J to compare is: }\PY{l+s+si}{\PYZob{}new\PYZus{}J\PYZcb{}}\PY{l+s+s1}{\PYZsq{}}\PY{p}{)}
        \PY{k}{if} \PY{n}{new\PYZus{}J} \PY{o}{\PYZlt{}} \PY{n}{lowest\PYZus{}J}\PY{p}{:}
            \PY{n}{lowest\PYZus{}J} \PY{o}{=} \PY{n}{new\PYZus{}J}
            \PY{n}{final\PYZus{}centroids} \PY{o}{=} \PY{n}{new\PYZus{}centroids}
            \PY{n}{final\PYZus{}idx} \PY{o}{=} \PY{n}{new\PYZus{}idx}
            
    \PY{k}{return} \PY{n}{final\PYZus{}centroids}\PY{p}{,} \PY{n}{final\PYZus{}idx}
            
\end{Verbatim}
\end{tcolorbox}
\pagebreak
    \hypertarget{using-the-k-means-to-answer-questions}{%
\section{Using the K-means to Answer
Questions}\label{using-the-k-means-to-answer-questions}}

    \hypertarget{parameters-to-pass-to-the-algorithm}{%
\subsection{Parameters to Pass to the
Algorithm}\label{parameters-to-pass-to-the-algorithm}}

    There are three parameter that must be set for the algorithm:


\begin{itemize}
	\item 
	\textbf{K}: Number of Clusters
\item
  \textbf{max\_iter}: the number of iterations, it sets the number that
  K-means algorithm will be ran (every time with a randomly chosen
  centroids) and the one with the lowest distortion error will be picked
  up as the final solution
\item
  \textbf{iner\_max\_iters}: the number of inner iterations, it sets the
  number that K-means algorithm tries to enhance the centroids until the
  centroids do not move much (with a fixed initial centroids)
\end{itemize}

Obviously, the whole dataset will be fed into our code, and we will
deivide all its entries by 255 to make them less than one.

    \begin{tcolorbox}[breakable, size=fbox, boxrule=1pt, pad at break*=1mm,colback=cellbackground, colframe=cellborder]
\prompt{In}{incolor}{ }{\boxspacing}
\begin{Verbatim}[commandchars=\\\{\}]
\PY{n}{K} \PY{o}{=} \PY{l+m+mi}{20}
\PY{n}{max\PYZus{}iters} \PY{o}{=} \PY{l+m+mi}{10}
\PY{n}{iner\PYZus{}max\PYZus{}iters} \PY{o}{=} \PY{l+m+mi}{70}

\PY{n}{X} \PY{o}{=} \PY{n}{train\PYZus{}images\PYZus{}reshaped}\PY{o}{/}\PY{l+m+mi}{255}
\end{Verbatim}
\end{tcolorbox}

    \hypertarget{getting-final-centroids-and-clusters}{%
\subsection{Getting Final Centroids and
Clusters}\label{getting-final-centroids-and-clusters}}

    The main function call to get the final centroids and final clusters to
which the images are assigned. This function is called three times and
after that the centroids will be showed as images.

\begin{itemize}
\tightlist
\item
  \textbf{K=5}
\item
  \textbf{K=10}
\item
  \textbf{K=20}
\end{itemize}

    \begin{tcolorbox}[breakable, size=fbox, boxrule=1pt, pad at break*=1mm,colback=cellbackground, colframe=cellborder]
\prompt{In}{incolor}{ }{\boxspacing}
\begin{Verbatim}[commandchars=\\\{\}]
\PY{n}{final\PYZus{}centroids}\PY{p}{,} \PY{n}{final\PYZus{}idx} \PY{o}{=} \PY{n}{kmeans\PYZus{}algorithm\PYZus{}with\PYZus{}distortion}\PY{p}{(}\PY{n}{X}\PY{p}{,} \PY{n}{max\PYZus{}iters}\PY{p}{,} \PY{n}{iner\PYZus{}max\PYZus{}iters}\PY{p}{,} \PY{n}{K}\PY{p}{)}
\end{Verbatim}
\end{tcolorbox}

    \hypertarget{centroids-images-and-clusters}{%
\subsection{5 Centroids: Images and
Clusters}\label{centroids-images-and-clusters}}

    Centroids are shown as 28x28 images below from left to right with
indixes 0 to 5:

    \begin{tcolorbox}[breakable, size=fbox, boxrule=1pt, pad at break*=1mm,colback=cellbackground, colframe=cellborder]
\prompt{In}{incolor}{ }{\boxspacing}
\begin{Verbatim}[commandchars=\\\{\}]
\PY{c+c1}{\PYZsh{} K=5, max\PYZus{}iters=15, max\PYZus{}inner\PYZus{}iters=40}
\PY{n}{show\PYZus{}centroids}\PY{p}{(}\PY{n}{final\PYZus{}centroids}\PY{p}{,} \PY{n}{K}\PY{p}{)}
\end{Verbatim}
\end{tcolorbox}

    \begin{center}
    \adjustimage{max size={0.9\linewidth}{0.9\paperheight}}{output_24_0.png}
    \end{center}
    { \hspace*{\fill} \\}
    
    Number of images assigned to each centroids, namely clusters:

    \begin{tcolorbox}[breakable, size=fbox, boxrule=1pt, pad at break*=1mm,colback=cellbackground, colframe=cellborder]
\prompt{In}{incolor}{ }{\boxspacing}
\begin{Verbatim}[commandchars=\\\{\}]
\PY{k}{for} \PY{n}{i} \PY{o+ow}{in} \PY{n}{np}\PY{o}{.}\PY{n}{unique}\PY{p}{(}\PY{n}{final\PYZus{}idx}\PY{p}{)}\PY{o}{.}\PY{n}{astype}\PY{p}{(}\PY{n+nb}{int}\PY{p}{)}\PY{p}{:}
    \PY{n+nb}{print}\PY{p}{(}\PY{l+s+sa}{f}\PY{l+s+s1}{\PYZsq{}}\PY{l+s+s1}{number of input in idx=}\PY{l+s+si}{\PYZob{}i\PYZcb{}}\PY{l+s+s1}{ cluster: }\PY{l+s+s1}{\PYZsq{}}\PY{p}{,} \PY{n}{X}\PY{p}{[}\PY{n}{final\PYZus{}idx}\PY{o}{.}\PY{n}{astype}\PY{p}{(}\PY{n+nb}{int}\PY{p}{)} \PY{o}{==} \PY{n}{i}\PY{p}{,} \PY{p}{:}\PY{p}{]}\PY{o}{.}\PY{n}{shape}\PY{p}{[}\PY{l+m+mi}{0}\PY{p}{]}\PY{p}{)}
\end{Verbatim}
\end{tcolorbox}

    \begin{Verbatim}[commandchars=\\\{\}]
number of input in idx=0 cluster:  11562
number of input in idx=1 cluster:  10252
number of input in idx=2 cluster:  13071
number of input in idx=3 cluster:  13433
number of input in idx=4 cluster:  11682
    \end{Verbatim}

    \hypertarget{centroids-images-and-clusters}{%
\subsection{10 Centroids: Images and
Clusters}\label{centroids-images-and-clusters}}

    Centroids are shown as 28x28 images below from left to right with
indixes 0 to 10:

    \begin{tcolorbox}[breakable, size=fbox, boxrule=1pt, pad at break*=1mm,colback=cellbackground, colframe=cellborder]
\prompt{In}{incolor}{ }{\boxspacing}
\begin{Verbatim}[commandchars=\\\{\}]
\PY{c+c1}{\PYZsh{} K=10, max\PYZus{}iters=10, max\PYZus{}inner\PYZus{}iters=50}
\PY{n}{show\PYZus{}centroids}\PY{p}{(}\PY{n}{final\PYZus{}centroids}\PY{p}{,} \PY{n}{K}\PY{p}{)}
\end{Verbatim}
\end{tcolorbox}

    \begin{center}
    \adjustimage{max size={0.9\linewidth}{0.9\paperheight}}{output_29_0.png}
    \end{center}
    { \hspace*{\fill} \\}
    
    Number of images assigned to each centroids, namely clusters:

    \begin{tcolorbox}[breakable, size=fbox, boxrule=1pt, pad at break*=1mm,colback=cellbackground, colframe=cellborder]
\prompt{In}{incolor}{ }{\boxspacing}
\begin{Verbatim}[commandchars=\\\{\}]
\PY{k}{for} \PY{n}{i} \PY{o+ow}{in} \PY{n}{np}\PY{o}{.}\PY{n}{unique}\PY{p}{(}\PY{n}{final\PYZus{}idx}\PY{p}{)}\PY{o}{.}\PY{n}{astype}\PY{p}{(}\PY{n+nb}{int}\PY{p}{)}\PY{p}{:}
    \PY{n+nb}{print}\PY{p}{(}\PY{l+s+sa}{f}\PY{l+s+s1}{\PYZsq{}}\PY{l+s+s1}{number of input in idx=}\PY{l+s+si}{\PYZob{}i\PYZcb{}}\PY{l+s+s1}{ cluster: }\PY{l+s+s1}{\PYZsq{}}\PY{p}{,} \PY{n}{X}\PY{p}{[}\PY{n}{final\PYZus{}idx}\PY{o}{.}\PY{n}{astype}\PY{p}{(}\PY{n+nb}{int}\PY{p}{)} \PY{o}{==} \PY{n}{i}\PY{p}{,} \PY{p}{:}\PY{p}{]}\PY{o}{.}\PY{n}{shape}\PY{p}{[}\PY{l+m+mi}{0}\PY{p}{]}\PY{p}{)}
\end{Verbatim}
\end{tcolorbox}

    \begin{Verbatim}[commandchars=\\\{\}]
number of input in idx=0 cluster:  6059
number of input in idx=1 cluster:  6459
number of input in idx=2 cluster:  6159
number of input in idx=3 cluster:  7621
number of input in idx=4 cluster:  5298
number of input in idx=5 cluster:  5196
number of input in idx=6 cluster:  4705
number of input in idx=7 cluster:  6660
number of input in idx=8 cluster:  6941
number of input in idx=9 cluster:  4902
    \end{Verbatim}

    \hypertarget{centroids-images-and-clusters}{%
\subsection{20 Centroids: Images and
Clusters}\label{centroids-images-and-clusters}}

    Centroids are shown as 28x28 images below from left to right with
indixes 0 to 5:

    \begin{tcolorbox}[breakable, size=fbox, boxrule=1pt, pad at break*=1mm,colback=cellbackground, colframe=cellborder]
\prompt{In}{incolor}{8}{\boxspacing}
\begin{Verbatim}[commandchars=\\\{\}]
\PY{c+c1}{\PYZsh{} K=20, max\PYZus{}iters=10, max\PYZus{}inner\PYZus{}iters=70}
\PY{n}{show\PYZus{}centroids}\PY{p}{(}\PY{n}{final\PYZus{}centroids}\PY{p}{,} \PY{n}{K}\PY{p}{)}
\end{Verbatim}
\end{tcolorbox}

    \begin{center}
    \adjustimage{max size={0.9\linewidth}{0.9\paperheight}}{output_34_0.png}
    \end{center}
    { \hspace*{\fill} \\}
    
    Number of images assigned to each centroids, namely clusters:

    \begin{tcolorbox}[breakable, size=fbox, boxrule=1pt, pad at break*=1mm,colback=cellbackground, colframe=cellborder]
\prompt{In}{incolor}{9}{\boxspacing}
\begin{Verbatim}[commandchars=\\\{\}]
\PY{k}{for} \PY{n}{i} \PY{o+ow}{in} \PY{n}{np}\PY{o}{.}\PY{n}{unique}\PY{p}{(}\PY{n}{final\PYZus{}idx}\PY{p}{)}\PY{o}{.}\PY{n}{astype}\PY{p}{(}\PY{n+nb}{int}\PY{p}{)}\PY{p}{:}
    \PY{n+nb}{print}\PY{p}{(}\PY{l+s+sa}{f}\PY{l+s+s1}{\PYZsq{}}\PY{l+s+s1}{number of input in idx=}\PY{l+s+si}{\PYZob{}i\PYZcb{}}\PY{l+s+s1}{ cluster: }\PY{l+s+s1}{\PYZsq{}}\PY{p}{,} \PY{n}{X}\PY{p}{[}\PY{n}{final\PYZus{}idx}\PY{o}{.}\PY{n}{astype}\PY{p}{(}\PY{n+nb}{int}\PY{p}{)} \PY{o}{==} \PY{n}{i}\PY{p}{,} \PY{p}{:}\PY{p}{]}\PY{o}{.}\PY{n}{shape}\PY{p}{[}\PY{l+m+mi}{0}\PY{p}{]}\PY{p}{)}
\end{Verbatim}
\end{tcolorbox}

    \begin{Verbatim}[commandchars=\\\{\}]
number of input in idx=0 cluster:  2265
number of input in idx=1 cluster:  2507
number of input in idx=2 cluster:  2685
number of input in idx=3 cluster:  2916
number of input in idx=4 cluster:  2793
number of input in idx=5 cluster:  3026
number of input in idx=6 cluster:  2622
number of input in idx=7 cluster:  3083
number of input in idx=8 cluster:  3082
number of input in idx=9 cluster:  4255
number of input in idx=10 cluster:  4332
number of input in idx=11 cluster:  4250
number of input in idx=12 cluster:  2651
number of input in idx=13 cluster:  2156
number of input in idx=14 cluster:  2637
number of input in idx=15 cluster:  4694
number of input in idx=16 cluster:  1514
number of input in idx=17 cluster:  2454
number of input in idx=18 cluster:  3369
number of input in idx=19 cluster:  2709
    \end{Verbatim}
\pagebreak
    \hypertarget{more-comments-and-comparison-with-global-pca}{%
\section{More Comments and Comparison with Global
PCA}\label{more-comments-and-comparison-with-global-pca}}

    \begin{itemize}
\item
  \textbf{K=5}

  It can be seen from section \textbf{3.3} that the number of clusters
  is limited and the centroids are like each other. They result
  in \emph{5} approximately equal distributed clusters.
\item
  \textbf{K=10}

  As we can see from section \textbf{3.4}, when the number of centroids
  is fixed on 10, they are very similar to the original digits \emph{0}
  to \emph{9} although there is two centroids that are like digit
  \emph{1} and unfortunately it has not predicted digit \emph{5}. As it
  can be seen from their images, we could say that omitting the digit
  \emph{5} is the result of its resemblance to digit \emph{2}. In the previous homework (HW3), when we did a global PCA on this same dataset, we printed the first 20 eigenvectors as images. Between them there were also some digits visible such as digits 0, 7, 3, 6, 8 but they were not very clear in comparison to the centroids we have found here. This conclusion seems to be reasonable because the centroids are actually the mean over the same digits, so when we plot them, they will look like actual digits. But the eigenvectors are not like this. Eignevectors try to keep the part of matrix in which most of the information exists, and there is no reason for them to be like an actual image of a digit.  
\item
  \textbf{K=20}

  When the number of centroids was increased to 20 as it is in section
  \textbf{3.5}, we have more free space to detect different digits. So
  in 20 centroids we could see all the digits from \emph{0} to \emph{9}
  and obviously some of them will be repeated.
\end{itemize}


    % Add a bibliography block to the postdoc
    
    
    
\end{document}
