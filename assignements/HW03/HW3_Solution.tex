\documentclass[11pt]{article}

\newcommand{\hmwkTitle}{HW\ \#3}
\newcommand{\hmwkDueDate}{}
\newcommand{\hmwkClass}{CpE 520}
\newcommand{\institute}{West Virginia University}
\newcommand{\hmwkClassInstructor}{Professor Nasser Nasrabadi}
\newcommand{\hmwkAuthorName}{\textbf{Ali Zafari}}

\title{
	\vspace{2in}
	\textmd{\textbf{\hmwkClass:\ \hmwkTitle}}\\
	\vspace{0.1in}\large{\institute}\\
%	\vspace{0.1in}\large{\textit{\hmwkClassInstructor}}
	\vspace{3in}
}

\author{\hmwkAuthorName}
\date{}






    \usepackage[breakable]{tcolorbox}
    \usepackage{parskip} % Stop auto-indenting (to mimic markdown behaviour)
    
    \usepackage{iftex}
    \ifPDFTeX
    	\usepackage[T1]{fontenc}
    	\usepackage{mathpazo}
    \else
    	\usepackage{fontspec}
    \fi

    % Basic figure setup, for now with no caption control since it's done
    % automatically by Pandoc (which extracts ![](path) syntax from Markdown).
    \usepackage{graphicx}
    % Maintain compatibility with old templates. Remove in nbconvert 6.0
    \let\Oldincludegraphics\includegraphics
    % Ensure that by default, figures have no caption (until we provide a
    % proper Figure object with a Caption API and a way to capture that
    % in the conversion process - todo).
    \usepackage{caption}
    \DeclareCaptionFormat{nocaption}{}
    \captionsetup{format=nocaption,aboveskip=0pt,belowskip=0pt}

    \usepackage[Export]{adjustbox} % Used to constrain images to a maximum size
    \adjustboxset{max size={0.9\linewidth}{0.9\paperheight}}
    \usepackage{float}
    \floatplacement{figure}{H} % forces figures to be placed at the correct location
    \usepackage{xcolor} % Allow colors to be defined
    \usepackage{enumerate} % Needed for markdown enumerations to work
    \usepackage{geometry} % Used to adjust the document margins
    \usepackage{amsmath} % Equations
    \usepackage{amssymb} % Equations
    \usepackage{textcomp} % defines textquotesingle
    % Hack from http://tex.stackexchange.com/a/47451/13684:
    \AtBeginDocument{%
        \def\PYZsq{\textquotesingle}% Upright quotes in Pygmentized code
    }
    \usepackage{upquote} % Upright quotes for verbatim code
    \usepackage{eurosym} % defines \euro
    \usepackage[mathletters]{ucs} % Extended unicode (utf-8) support
    \usepackage{fancyvrb} % verbatim replacement that allows latex
    \usepackage{grffile} % extends the file name processing of package graphics 
                         % to support a larger range
    \makeatletter % fix for grffile with XeLaTeX
    \def\Gread@@xetex#1{%
      \IfFileExists{"\Gin@base".bb}%
      {\Gread@eps{\Gin@base.bb}}%
      {\Gread@@xetex@aux#1}%
    }
    \makeatother

    % The hyperref package gives us a pdf with properly built
    % internal navigation ('pdf bookmarks' for the table of contents,
    % internal cross-reference links, web links for URLs, etc.)
    \usepackage{hyperref}
    % The default LaTeX title has an obnoxious amount of whitespace. By default,
    % titling removes some of it. It also provides customization options.
    \usepackage{titling}
    \usepackage{longtable} % longtable support required by pandoc >1.10
    \usepackage{booktabs}  % table support for pandoc > 1.12.2
    \usepackage[inline]{enumitem} % IRkernel/repr support (it uses the enumerate* environment)
    \usepackage[normalem]{ulem} % ulem is needed to support strikethroughs (\sout)
                                % normalem makes italics be italics, not underlines
    \usepackage{mathrsfs}
    

    
    % Colors for the hyperref package
    \definecolor{urlcolor}{rgb}{0,.145,.698}
    \definecolor{linkcolor}{rgb}{.71,0.21,0.01}
    \definecolor{citecolor}{rgb}{.12,.54,.11}

    % ANSI colors
    \definecolor{ansi-black}{HTML}{3E424D}
    \definecolor{ansi-black-intense}{HTML}{282C36}
    \definecolor{ansi-red}{HTML}{E75C58}
    \definecolor{ansi-red-intense}{HTML}{B22B31}
    \definecolor{ansi-green}{HTML}{00A250}
    \definecolor{ansi-green-intense}{HTML}{007427}
    \definecolor{ansi-yellow}{HTML}{DDB62B}
    \definecolor{ansi-yellow-intense}{HTML}{B27D12}
    \definecolor{ansi-blue}{HTML}{208FFB}
    \definecolor{ansi-blue-intense}{HTML}{0065CA}
    \definecolor{ansi-magenta}{HTML}{D160C4}
    \definecolor{ansi-magenta-intense}{HTML}{A03196}
    \definecolor{ansi-cyan}{HTML}{60C6C8}
    \definecolor{ansi-cyan-intense}{HTML}{258F8F}
    \definecolor{ansi-white}{HTML}{C5C1B4}
    \definecolor{ansi-white-intense}{HTML}{A1A6B2}
    \definecolor{ansi-default-inverse-fg}{HTML}{FFFFFF}
    \definecolor{ansi-default-inverse-bg}{HTML}{000000}

    % commands and environments needed by pandoc snippets
    % extracted from the output of `pandoc -s`
    \providecommand{\tightlist}{%
      \setlength{\itemsep}{0pt}\setlength{\parskip}{0pt}}
    \DefineVerbatimEnvironment{Highlighting}{Verbatim}{commandchars=\\\{\}}
    % Add ',fontsize=\small' for more characters per line
    \newenvironment{Shaded}{}{}
    \newcommand{\KeywordTok}[1]{\textcolor[rgb]{0.00,0.44,0.13}{\textbf{{#1}}}}
    \newcommand{\DataTypeTok}[1]{\textcolor[rgb]{0.56,0.13,0.00}{{#1}}}
    \newcommand{\DecValTok}[1]{\textcolor[rgb]{0.25,0.63,0.44}{{#1}}}
    \newcommand{\BaseNTok}[1]{\textcolor[rgb]{0.25,0.63,0.44}{{#1}}}
    \newcommand{\FloatTok}[1]{\textcolor[rgb]{0.25,0.63,0.44}{{#1}}}
    \newcommand{\CharTok}[1]{\textcolor[rgb]{0.25,0.44,0.63}{{#1}}}
    \newcommand{\StringTok}[1]{\textcolor[rgb]{0.25,0.44,0.63}{{#1}}}
    \newcommand{\CommentTok}[1]{\textcolor[rgb]{0.38,0.63,0.69}{\textit{{#1}}}}
    \newcommand{\OtherTok}[1]{\textcolor[rgb]{0.00,0.44,0.13}{{#1}}}
    \newcommand{\AlertTok}[1]{\textcolor[rgb]{1.00,0.00,0.00}{\textbf{{#1}}}}
    \newcommand{\FunctionTok}[1]{\textcolor[rgb]{0.02,0.16,0.49}{{#1}}}
    \newcommand{\RegionMarkerTok}[1]{{#1}}
    \newcommand{\ErrorTok}[1]{\textcolor[rgb]{1.00,0.00,0.00}{\textbf{{#1}}}}
    \newcommand{\NormalTok}[1]{{#1}}
    
    % Additional commands for more recent versions of Pandoc
    \newcommand{\ConstantTok}[1]{\textcolor[rgb]{0.53,0.00,0.00}{{#1}}}
    \newcommand{\SpecialCharTok}[1]{\textcolor[rgb]{0.25,0.44,0.63}{{#1}}}
    \newcommand{\VerbatimStringTok}[1]{\textcolor[rgb]{0.25,0.44,0.63}{{#1}}}
    \newcommand{\SpecialStringTok}[1]{\textcolor[rgb]{0.73,0.40,0.53}{{#1}}}
    \newcommand{\ImportTok}[1]{{#1}}
    \newcommand{\DocumentationTok}[1]{\textcolor[rgb]{0.73,0.13,0.13}{\textit{{#1}}}}
    \newcommand{\AnnotationTok}[1]{\textcolor[rgb]{0.38,0.63,0.69}{\textbf{\textit{{#1}}}}}
    \newcommand{\CommentVarTok}[1]{\textcolor[rgb]{0.38,0.63,0.69}{\textbf{\textit{{#1}}}}}
    \newcommand{\VariableTok}[1]{\textcolor[rgb]{0.10,0.09,0.49}{{#1}}}
    \newcommand{\ControlFlowTok}[1]{\textcolor[rgb]{0.00,0.44,0.13}{\textbf{{#1}}}}
    \newcommand{\OperatorTok}[1]{\textcolor[rgb]{0.40,0.40,0.40}{{#1}}}
    \newcommand{\BuiltInTok}[1]{{#1}}
    \newcommand{\ExtensionTok}[1]{{#1}}
    \newcommand{\PreprocessorTok}[1]{\textcolor[rgb]{0.74,0.48,0.00}{{#1}}}
    \newcommand{\AttributeTok}[1]{\textcolor[rgb]{0.49,0.56,0.16}{{#1}}}
    \newcommand{\InformationTok}[1]{\textcolor[rgb]{0.38,0.63,0.69}{\textbf{\textit{{#1}}}}}
    \newcommand{\WarningTok}[1]{\textcolor[rgb]{0.38,0.63,0.69}{\textbf{\textit{{#1}}}}}
    
    
    % Define a nice break command that doesn't care if a line doesn't already
    % exist.
    \def\br{\hspace*{\fill} \\* }
    % Math Jax compatibility definitions
    \def\gt{>}
    \def\lt{<}
    \let\Oldtex\TeX
    \let\Oldlatex\LaTeX
    \renewcommand{\TeX}{\textrm{\Oldtex}}
    \renewcommand{\LaTeX}{\textrm{\Oldlatex}}
    % Document parameters
    % Document title
    %\title{HW3}
    
    
    
    
    
% Pygments definitions
\makeatletter
\def\PY@reset{\let\PY@it=\relax \let\PY@bf=\relax%
    \let\PY@ul=\relax \let\PY@tc=\relax%
    \let\PY@bc=\relax \let\PY@ff=\relax}
\def\PY@tok#1{\csname PY@tok@#1\endcsname}
\def\PY@toks#1+{\ifx\relax#1\empty\else%
    \PY@tok{#1}\expandafter\PY@toks\fi}
\def\PY@do#1{\PY@bc{\PY@tc{\PY@ul{%
    \PY@it{\PY@bf{\PY@ff{#1}}}}}}}
\def\PY#1#2{\PY@reset\PY@toks#1+\relax+\PY@do{#2}}

\expandafter\def\csname PY@tok@w\endcsname{\def\PY@tc##1{\textcolor[rgb]{0.73,0.73,0.73}{##1}}}
\expandafter\def\csname PY@tok@c\endcsname{\let\PY@it=\textit\def\PY@tc##1{\textcolor[rgb]{0.25,0.50,0.50}{##1}}}
\expandafter\def\csname PY@tok@cp\endcsname{\def\PY@tc##1{\textcolor[rgb]{0.74,0.48,0.00}{##1}}}
\expandafter\def\csname PY@tok@k\endcsname{\let\PY@bf=\textbf\def\PY@tc##1{\textcolor[rgb]{0.00,0.50,0.00}{##1}}}
\expandafter\def\csname PY@tok@kp\endcsname{\def\PY@tc##1{\textcolor[rgb]{0.00,0.50,0.00}{##1}}}
\expandafter\def\csname PY@tok@kt\endcsname{\def\PY@tc##1{\textcolor[rgb]{0.69,0.00,0.25}{##1}}}
\expandafter\def\csname PY@tok@o\endcsname{\def\PY@tc##1{\textcolor[rgb]{0.40,0.40,0.40}{##1}}}
\expandafter\def\csname PY@tok@ow\endcsname{\let\PY@bf=\textbf\def\PY@tc##1{\textcolor[rgb]{0.67,0.13,1.00}{##1}}}
\expandafter\def\csname PY@tok@nb\endcsname{\def\PY@tc##1{\textcolor[rgb]{0.00,0.50,0.00}{##1}}}
\expandafter\def\csname PY@tok@nf\endcsname{\def\PY@tc##1{\textcolor[rgb]{0.00,0.00,1.00}{##1}}}
\expandafter\def\csname PY@tok@nc\endcsname{\let\PY@bf=\textbf\def\PY@tc##1{\textcolor[rgb]{0.00,0.00,1.00}{##1}}}
\expandafter\def\csname PY@tok@nn\endcsname{\let\PY@bf=\textbf\def\PY@tc##1{\textcolor[rgb]{0.00,0.00,1.00}{##1}}}
\expandafter\def\csname PY@tok@ne\endcsname{\let\PY@bf=\textbf\def\PY@tc##1{\textcolor[rgb]{0.82,0.25,0.23}{##1}}}
\expandafter\def\csname PY@tok@nv\endcsname{\def\PY@tc##1{\textcolor[rgb]{0.10,0.09,0.49}{##1}}}
\expandafter\def\csname PY@tok@no\endcsname{\def\PY@tc##1{\textcolor[rgb]{0.53,0.00,0.00}{##1}}}
\expandafter\def\csname PY@tok@nl\endcsname{\def\PY@tc##1{\textcolor[rgb]{0.63,0.63,0.00}{##1}}}
\expandafter\def\csname PY@tok@ni\endcsname{\let\PY@bf=\textbf\def\PY@tc##1{\textcolor[rgb]{0.60,0.60,0.60}{##1}}}
\expandafter\def\csname PY@tok@na\endcsname{\def\PY@tc##1{\textcolor[rgb]{0.49,0.56,0.16}{##1}}}
\expandafter\def\csname PY@tok@nt\endcsname{\let\PY@bf=\textbf\def\PY@tc##1{\textcolor[rgb]{0.00,0.50,0.00}{##1}}}
\expandafter\def\csname PY@tok@nd\endcsname{\def\PY@tc##1{\textcolor[rgb]{0.67,0.13,1.00}{##1}}}
\expandafter\def\csname PY@tok@s\endcsname{\def\PY@tc##1{\textcolor[rgb]{0.73,0.13,0.13}{##1}}}
\expandafter\def\csname PY@tok@sd\endcsname{\let\PY@it=\textit\def\PY@tc##1{\textcolor[rgb]{0.73,0.13,0.13}{##1}}}
\expandafter\def\csname PY@tok@si\endcsname{\let\PY@bf=\textbf\def\PY@tc##1{\textcolor[rgb]{0.73,0.40,0.53}{##1}}}
\expandafter\def\csname PY@tok@se\endcsname{\let\PY@bf=\textbf\def\PY@tc##1{\textcolor[rgb]{0.73,0.40,0.13}{##1}}}
\expandafter\def\csname PY@tok@sr\endcsname{\def\PY@tc##1{\textcolor[rgb]{0.73,0.40,0.53}{##1}}}
\expandafter\def\csname PY@tok@ss\endcsname{\def\PY@tc##1{\textcolor[rgb]{0.10,0.09,0.49}{##1}}}
\expandafter\def\csname PY@tok@sx\endcsname{\def\PY@tc##1{\textcolor[rgb]{0.00,0.50,0.00}{##1}}}
\expandafter\def\csname PY@tok@m\endcsname{\def\PY@tc##1{\textcolor[rgb]{0.40,0.40,0.40}{##1}}}
\expandafter\def\csname PY@tok@gh\endcsname{\let\PY@bf=\textbf\def\PY@tc##1{\textcolor[rgb]{0.00,0.00,0.50}{##1}}}
\expandafter\def\csname PY@tok@gu\endcsname{\let\PY@bf=\textbf\def\PY@tc##1{\textcolor[rgb]{0.50,0.00,0.50}{##1}}}
\expandafter\def\csname PY@tok@gd\endcsname{\def\PY@tc##1{\textcolor[rgb]{0.63,0.00,0.00}{##1}}}
\expandafter\def\csname PY@tok@gi\endcsname{\def\PY@tc##1{\textcolor[rgb]{0.00,0.63,0.00}{##1}}}
\expandafter\def\csname PY@tok@gr\endcsname{\def\PY@tc##1{\textcolor[rgb]{1.00,0.00,0.00}{##1}}}
\expandafter\def\csname PY@tok@ge\endcsname{\let\PY@it=\textit}
\expandafter\def\csname PY@tok@gs\endcsname{\let\PY@bf=\textbf}
\expandafter\def\csname PY@tok@gp\endcsname{\let\PY@bf=\textbf\def\PY@tc##1{\textcolor[rgb]{0.00,0.00,0.50}{##1}}}
\expandafter\def\csname PY@tok@go\endcsname{\def\PY@tc##1{\textcolor[rgb]{0.53,0.53,0.53}{##1}}}
\expandafter\def\csname PY@tok@gt\endcsname{\def\PY@tc##1{\textcolor[rgb]{0.00,0.27,0.87}{##1}}}
\expandafter\def\csname PY@tok@err\endcsname{\def\PY@bc##1{\setlength{\fboxsep}{0pt}\fcolorbox[rgb]{1.00,0.00,0.00}{1,1,1}{\strut ##1}}}
\expandafter\def\csname PY@tok@kc\endcsname{\let\PY@bf=\textbf\def\PY@tc##1{\textcolor[rgb]{0.00,0.50,0.00}{##1}}}
\expandafter\def\csname PY@tok@kd\endcsname{\let\PY@bf=\textbf\def\PY@tc##1{\textcolor[rgb]{0.00,0.50,0.00}{##1}}}
\expandafter\def\csname PY@tok@kn\endcsname{\let\PY@bf=\textbf\def\PY@tc##1{\textcolor[rgb]{0.00,0.50,0.00}{##1}}}
\expandafter\def\csname PY@tok@kr\endcsname{\let\PY@bf=\textbf\def\PY@tc##1{\textcolor[rgb]{0.00,0.50,0.00}{##1}}}
\expandafter\def\csname PY@tok@bp\endcsname{\def\PY@tc##1{\textcolor[rgb]{0.00,0.50,0.00}{##1}}}
\expandafter\def\csname PY@tok@fm\endcsname{\def\PY@tc##1{\textcolor[rgb]{0.00,0.00,1.00}{##1}}}
\expandafter\def\csname PY@tok@vc\endcsname{\def\PY@tc##1{\textcolor[rgb]{0.10,0.09,0.49}{##1}}}
\expandafter\def\csname PY@tok@vg\endcsname{\def\PY@tc##1{\textcolor[rgb]{0.10,0.09,0.49}{##1}}}
\expandafter\def\csname PY@tok@vi\endcsname{\def\PY@tc##1{\textcolor[rgb]{0.10,0.09,0.49}{##1}}}
\expandafter\def\csname PY@tok@vm\endcsname{\def\PY@tc##1{\textcolor[rgb]{0.10,0.09,0.49}{##1}}}
\expandafter\def\csname PY@tok@sa\endcsname{\def\PY@tc##1{\textcolor[rgb]{0.73,0.13,0.13}{##1}}}
\expandafter\def\csname PY@tok@sb\endcsname{\def\PY@tc##1{\textcolor[rgb]{0.73,0.13,0.13}{##1}}}
\expandafter\def\csname PY@tok@sc\endcsname{\def\PY@tc##1{\textcolor[rgb]{0.73,0.13,0.13}{##1}}}
\expandafter\def\csname PY@tok@dl\endcsname{\def\PY@tc##1{\textcolor[rgb]{0.73,0.13,0.13}{##1}}}
\expandafter\def\csname PY@tok@s2\endcsname{\def\PY@tc##1{\textcolor[rgb]{0.73,0.13,0.13}{##1}}}
\expandafter\def\csname PY@tok@sh\endcsname{\def\PY@tc##1{\textcolor[rgb]{0.73,0.13,0.13}{##1}}}
\expandafter\def\csname PY@tok@s1\endcsname{\def\PY@tc##1{\textcolor[rgb]{0.73,0.13,0.13}{##1}}}
\expandafter\def\csname PY@tok@mb\endcsname{\def\PY@tc##1{\textcolor[rgb]{0.40,0.40,0.40}{##1}}}
\expandafter\def\csname PY@tok@mf\endcsname{\def\PY@tc##1{\textcolor[rgb]{0.40,0.40,0.40}{##1}}}
\expandafter\def\csname PY@tok@mh\endcsname{\def\PY@tc##1{\textcolor[rgb]{0.40,0.40,0.40}{##1}}}
\expandafter\def\csname PY@tok@mi\endcsname{\def\PY@tc##1{\textcolor[rgb]{0.40,0.40,0.40}{##1}}}
\expandafter\def\csname PY@tok@il\endcsname{\def\PY@tc##1{\textcolor[rgb]{0.40,0.40,0.40}{##1}}}
\expandafter\def\csname PY@tok@mo\endcsname{\def\PY@tc##1{\textcolor[rgb]{0.40,0.40,0.40}{##1}}}
\expandafter\def\csname PY@tok@ch\endcsname{\let\PY@it=\textit\def\PY@tc##1{\textcolor[rgb]{0.25,0.50,0.50}{##1}}}
\expandafter\def\csname PY@tok@cm\endcsname{\let\PY@it=\textit\def\PY@tc##1{\textcolor[rgb]{0.25,0.50,0.50}{##1}}}
\expandafter\def\csname PY@tok@cpf\endcsname{\let\PY@it=\textit\def\PY@tc##1{\textcolor[rgb]{0.25,0.50,0.50}{##1}}}
\expandafter\def\csname PY@tok@c1\endcsname{\let\PY@it=\textit\def\PY@tc##1{\textcolor[rgb]{0.25,0.50,0.50}{##1}}}
\expandafter\def\csname PY@tok@cs\endcsname{\let\PY@it=\textit\def\PY@tc##1{\textcolor[rgb]{0.25,0.50,0.50}{##1}}}

\def\PYZbs{\char`\\}
\def\PYZus{\char`\_}
\def\PYZob{\char`\{}
\def\PYZcb{\char`\}}
\def\PYZca{\char`\^}
\def\PYZam{\char`\&}
\def\PYZlt{\char`\<}
\def\PYZgt{\char`\>}
\def\PYZsh{\char`\#}
\def\PYZpc{\char`\%}
\def\PYZdl{\char`\$}
\def\PYZhy{\char`\-}
\def\PYZsq{\char`\'}
\def\PYZdq{\char`\"}
\def\PYZti{\char`\~}
% for compatibility with earlier versions
\def\PYZat{@}
\def\PYZlb{[}
\def\PYZrb{]}
\makeatother


    % For linebreaks inside Verbatim environment from package fancyvrb. 
    \makeatletter
        \newbox\Wrappedcontinuationbox 
        \newbox\Wrappedvisiblespacebox 
        \newcommand*\Wrappedvisiblespace {\textcolor{red}{\textvisiblespace}} 
        \newcommand*\Wrappedcontinuationsymbol {\textcolor{red}{\llap{\tiny$\m@th\hookrightarrow$}}} 
        \newcommand*\Wrappedcontinuationindent {3ex } 
        \newcommand*\Wrappedafterbreak {\kern\Wrappedcontinuationindent\copy\Wrappedcontinuationbox} 
        % Take advantage of the already applied Pygments mark-up to insert 
        % potential linebreaks for TeX processing. 
        %        {, <, #, %, $, ' and ": go to next line. 
        %        _, }, ^, &, >, - and ~: stay at end of broken line. 
        % Use of \textquotesingle for straight quote. 
        \newcommand*\Wrappedbreaksatspecials {% 
            \def\PYGZus{\discretionary{\char`\_}{\Wrappedafterbreak}{\char`\_}}% 
            \def\PYGZob{\discretionary{}{\Wrappedafterbreak\char`\{}{\char`\{}}% 
            \def\PYGZcb{\discretionary{\char`\}}{\Wrappedafterbreak}{\char`\}}}% 
            \def\PYGZca{\discretionary{\char`\^}{\Wrappedafterbreak}{\char`\^}}% 
            \def\PYGZam{\discretionary{\char`\&}{\Wrappedafterbreak}{\char`\&}}% 
            \def\PYGZlt{\discretionary{}{\Wrappedafterbreak\char`\<}{\char`\<}}% 
            \def\PYGZgt{\discretionary{\char`\>}{\Wrappedafterbreak}{\char`\>}}% 
            \def\PYGZsh{\discretionary{}{\Wrappedafterbreak\char`\#}{\char`\#}}% 
            \def\PYGZpc{\discretionary{}{\Wrappedafterbreak\char`\%}{\char`\%}}% 
            \def\PYGZdl{\discretionary{}{\Wrappedafterbreak\char`\$}{\char`\$}}% 
            \def\PYGZhy{\discretionary{\char`\-}{\Wrappedafterbreak}{\char`\-}}% 
            \def\PYGZsq{\discretionary{}{\Wrappedafterbreak\textquotesingle}{\textquotesingle}}% 
            \def\PYGZdq{\discretionary{}{\Wrappedafterbreak\char`\"}{\char`\"}}% 
            \def\PYGZti{\discretionary{\char`\~}{\Wrappedafterbreak}{\char`\~}}% 
        } 
        % Some characters . , ; ? ! / are not pygmentized. 
        % This macro makes them "active" and they will insert potential linebreaks 
        \newcommand*\Wrappedbreaksatpunct {% 
            \lccode`\~`\.\lowercase{\def~}{\discretionary{\hbox{\char`\.}}{\Wrappedafterbreak}{\hbox{\char`\.}}}% 
            \lccode`\~`\,\lowercase{\def~}{\discretionary{\hbox{\char`\,}}{\Wrappedafterbreak}{\hbox{\char`\,}}}% 
            \lccode`\~`\;\lowercase{\def~}{\discretionary{\hbox{\char`\;}}{\Wrappedafterbreak}{\hbox{\char`\;}}}% 
            \lccode`\~`\:\lowercase{\def~}{\discretionary{\hbox{\char`\:}}{\Wrappedafterbreak}{\hbox{\char`\:}}}% 
            \lccode`\~`\?\lowercase{\def~}{\discretionary{\hbox{\char`\?}}{\Wrappedafterbreak}{\hbox{\char`\?}}}% 
            \lccode`\~`\!\lowercase{\def~}{\discretionary{\hbox{\char`\!}}{\Wrappedafterbreak}{\hbox{\char`\!}}}% 
            \lccode`\~`\/\lowercase{\def~}{\discretionary{\hbox{\char`\/}}{\Wrappedafterbreak}{\hbox{\char`\/}}}% 
            \catcode`\.\active
            \catcode`\,\active 
            \catcode`\;\active
            \catcode`\:\active
            \catcode`\?\active
            \catcode`\!\active
            \catcode`\/\active 
            \lccode`\~`\~ 	
        }
    \makeatother

    \let\OriginalVerbatim=\Verbatim
    \makeatletter
    \renewcommand{\Verbatim}[1][1]{%
        %\parskip\z@skip
        \sbox\Wrappedcontinuationbox {\Wrappedcontinuationsymbol}%
        \sbox\Wrappedvisiblespacebox {\FV@SetupFont\Wrappedvisiblespace}%
        \def\FancyVerbFormatLine ##1{\hsize\linewidth
            \vtop{\raggedright\hyphenpenalty\z@\exhyphenpenalty\z@
                \doublehyphendemerits\z@\finalhyphendemerits\z@
                \strut ##1\strut}%
        }%
        % If the linebreak is at a space, the latter will be displayed as visible
        % space at end of first line, and a continuation symbol starts next line.
        % Stretch/shrink are however usually zero for typewriter font.
        \def\FV@Space {%
            \nobreak\hskip\z@ plus\fontdimen3\font minus\fontdimen4\font
            \discretionary{\copy\Wrappedvisiblespacebox}{\Wrappedafterbreak}
            {\kern\fontdimen2\font}%
        }%
        
        % Allow breaks at special characters using \PYG... macros.
        \Wrappedbreaksatspecials
        % Breaks at punctuation characters . , ; ? ! and / need catcode=\active 	
        \OriginalVerbatim[#1,codes*=\Wrappedbreaksatpunct]%
    }
    \makeatother

    % Exact colors from NB
    \definecolor{incolor}{HTML}{303F9F}
    \definecolor{outcolor}{HTML}{D84315}
    \definecolor{cellborder}{HTML}{CFCFCF}
    \definecolor{cellbackground}{HTML}{F7F7F7}
    
    % prompt
    \makeatletter
    \newcommand{\boxspacing}{\kern\kvtcb@left@rule\kern\kvtcb@boxsep}
    \makeatother
    \newcommand{\prompt}[4]{
        \ttfamily\llap{{\color{#2}[#3]:\hspace{3pt}#4}}\vspace{-\baselineskip}
    }
    

    
    % Prevent overflowing lines due to hard-to-break entities
    \sloppy 
    % Setup hyperref package
    \hypersetup{
      breaklinks=true,  % so long urls are correctly broken across lines
      colorlinks=true,
      urlcolor=urlcolor,
      linkcolor=linkcolor,
      citecolor=citecolor,
      }
    % Slightly bigger margins than the latex defaults
    
    \geometry{verbose,tmargin=1in,bmargin=1in,lmargin=1in,rmargin=1in}
    
    

\begin{document}
	
	\pagenumbering{gobble}% prevent cover page of numbering
	\maketitle
	\pagebreak % let cover page free to the end of page
	\pagenumbering{arabic} % start page numbering again from 1 and print them!
	
    
    
    




\tableofcontents
\pagebreak

\section{Preface}
    At first, we need to import the required modules and of course the MNIST
dataset:

    \begin{tcolorbox}[breakable, size=fbox, boxrule=1pt, pad at break*=1mm,colback=cellbackground, colframe=cellborder]
\prompt{In}{incolor}{1}{\boxspacing}
\begin{Verbatim}[commandchars=\\\{\}]
\PY{k+kn}{import} \PY{n+nn}{mnist}
\PY{k+kn}{import} \PY{n+nn}{os}
\PY{k+kn}{import} \PY{n+nn}{numpy} \PY{k}{as} \PY{n+nn}{np}
\PY{k+kn}{import} \PY{n+nn}{matplotlib}\PY{n+nn}{.}\PY{n+nn}{pyplot} \PY{k}{as} \PY{n+nn}{plt}
\PY{k+kn}{from} \PY{n+nn}{sklearn}\PY{n+nn}{.}\PY{n+nn}{preprocessing} \PY{k+kn}{import} \PY{n}{StandardScaler}
\PY{k+kn}{from} \PY{n+nn}{scipy}\PY{n+nn}{.}\PY{n+nn}{linalg} \PY{k+kn}{import} \PY{n}{eigh}

\PY{n}{mnist}\PY{o}{.}\PY{n}{temporary\PYZus{}dir} \PY{o}{=} \PY{k}{lambda}\PY{p}{:} \PY{n}{os}\PY{o}{.}\PY{n}{getcwd}\PY{p}{(}\PY{p}{)}

\PY{c+c1}{\PYZsh{} Each of these functions first downloads the data and returns a numpy array.}
\PY{n}{train\PYZus{}images} \PY{o}{=} \PY{n}{mnist}\PY{o}{.}\PY{n}{train\PYZus{}images}\PY{p}{(}\PY{p}{)}
\PY{n}{train\PYZus{}labels} \PY{o}{=} \PY{n}{mnist}\PY{o}{.}\PY{n}{train\PYZus{}labels}\PY{p}{(}\PY{p}{)}

\PY{k}{assert} \PY{n}{train\PYZus{}images}\PY{o}{.}\PY{n}{shape} \PY{o}{==} \PY{p}{(}\PY{l+m+mi}{60000}\PY{p}{,} \PY{l+m+mi}{28}\PY{p}{,} \PY{l+m+mi}{28}\PY{p}{)}\PY{p}{,} \PY{l+s+s1}{\PYZsq{}}\PY{l+s+s1}{train\PYZus{}imgages shape is not correct}\PY{l+s+s1}{\PYZsq{}}
\PY{k}{assert} \PY{n}{train\PYZus{}labels}\PY{o}{.}\PY{n}{shape} \PY{o}{==} \PY{p}{(}\PY{l+m+mi}{60000}\PY{p}{,}\PY{p}{)}\PY{p}{,} \PY{l+s+s1}{\PYZsq{}}\PY{l+s+s1}{train\PYZus{}labels shape is not correct}\PY{l+s+s1}{\PYZsq{}}
\end{Verbatim}
\end{tcolorbox}

    Finding how many of every digit exists in the mnist dataset:

    \begin{tcolorbox}[breakable, size=fbox, boxrule=1pt, pad at break*=1mm,colback=cellbackground, colframe=cellborder]
\prompt{In}{incolor}{2}{\boxspacing}
\begin{Verbatim}[commandchars=\\\{\}]
\PY{n}{whole\PYZus{}number} \PY{o}{=} \PY{l+m+mi}{0}

\PY{k}{for} \PY{n}{i} \PY{o+ow}{in} \PY{n+nb}{range}\PY{p}{(}\PY{l+m+mi}{10}\PY{p}{)}\PY{p}{:}
    \PY{n}{number} \PY{o}{=} \PY{n}{train\PYZus{}images}\PY{p}{[}\PY{n}{train\PYZus{}labels} \PY{o}{==} \PY{n}{i}\PY{p}{,} \PY{p}{:}\PY{p}{]}\PY{o}{.}\PY{n}{shape}\PY{p}{[}\PY{l+m+mi}{0}\PY{p}{]}
    \PY{n}{whole\PYZus{}number} \PY{o}{+}\PY{o}{=} \PY{n}{number}
    \PY{n+nb}{print}\PY{p}{(}\PY{l+s+sa}{f}\PY{l+s+s1}{\PYZsq{}}\PY{l+s+s1}{number of digit }\PY{l+s+si}{\PYZob{}i\PYZcb{}}\PY{l+s+s1}{ is: }\PY{l+s+si}{\PYZob{}number\PYZcb{}}\PY{l+s+s1}{\PYZsq{}}\PY{p}{)}

\PY{k}{assert} \PY{n}{whole\PYZus{}number} \PY{o}{==} \PY{n}{train\PYZus{}images}\PY{o}{.}\PY{n}{shape}\PY{p}{[}\PY{l+m+mi}{0}\PY{p}{]}\PY{p}{,} \PY{l+s+s1}{\PYZsq{}}\PY{l+s+s1}{some data is missed}\PY{l+s+s1}{\PYZsq{}}
\end{Verbatim}
\end{tcolorbox}

    \begin{Verbatim}[commandchars=\\\{\}]
number of digit 0 is: 5923
number of digit 1 is: 6742
number of digit 2 is: 5958
number of digit 3 is: 6131
number of digit 4 is: 5842
number of digit 5 is: 5421
number of digit 6 is: 5918
number of digit 7 is: 6265
number of digit 8 is: 5851
number of digit 9 is: 5949
    \end{Verbatim}
    \pagebreak
	We show 25 samples of the dataset as an example below:
    \begin{tcolorbox}[breakable, size=fbox, boxrule=1pt, pad at break*=1mm,colback=cellbackground, colframe=cellborder]
\prompt{In}{incolor}{3}{\boxspacing}
\begin{Verbatim}[commandchars=\\\{\}]
\PY{n}{fig}\PY{p}{,} \PY{n}{ax} \PY{o}{=} \PY{n}{plt}\PY{o}{.}\PY{n}{subplots}\PY{p}{(}\PY{l+m+mi}{5}\PY{p}{,} \PY{l+m+mi}{5}\PY{p}{,} \PY{n}{figsize}\PY{o}{=}\PY{p}{(}\PY{l+m+mi}{20}\PY{p}{,}\PY{l+m+mi}{20}\PY{p}{)}\PY{p}{)}
\PY{k}{for} \PY{n}{i} \PY{o+ow}{in} \PY{n+nb}{range}\PY{p}{(}\PY{l+m+mi}{5}\PY{p}{)}\PY{p}{:}
    \PY{k}{for} \PY{n}{j} \PY{o+ow}{in} \PY{n+nb}{range}\PY{p}{(}\PY{l+m+mi}{5}\PY{p}{)}\PY{p}{:}
        \PY{n}{ax}\PY{p}{[}\PY{n}{i}\PY{p}{]}\PY{p}{[}\PY{n}{j}\PY{p}{]}\PY{o}{.}\PY{n}{imshow}\PY{p}{(}\PY{n}{train\PYZus{}images}\PY{p}{[}\PY{l+m+mi}{5} \PY{o}{*} \PY{n}{i} \PY{o}{+} \PY{n}{j}\PY{p}{]}\PY{p}{,} \PY{n}{cmap}\PY{o}{=}\PY{l+s+s1}{\PYZsq{}}\PY{l+s+s1}{Greys}\PY{l+s+s1}{\PYZsq{}}\PY{p}{)}
        \PY{n}{ax}\PY{p}{[}\PY{n}{i}\PY{p}{]}\PY{p}{[}\PY{n}{j}\PY{p}{]}\PY{o}{.}\PY{n}{xaxis}\PY{o}{.}\PY{n}{set\PYZus{}visible}\PY{p}{(}\PY{k+kc}{False}\PY{p}{)}
        \PY{n}{ax}\PY{p}{[}\PY{n}{i}\PY{p}{]}\PY{p}{[}\PY{n}{j}\PY{p}{]}\PY{o}{.}\PY{n}{yaxis}\PY{o}{.}\PY{n}{set\PYZus{}visible}\PY{p}{(}\PY{k+kc}{False}\PY{p}{)}
\PY{n}{plt}\PY{o}{.}\PY{n}{show}\PY{p}{(}\PY{p}{)}
\end{Verbatim}
\end{tcolorbox}

    \begin{center}
    \adjustimage{max size={0.6\linewidth}{0.6\paperheight}}{output_5_0.png}
    \end{center}
    { \hspace*{\fill} \\}
    
    We reshape the data from (60000, 28, 28) to (60000, 784), to make it
easier for calculations through the rest of project:

    \begin{tcolorbox}[breakable, size=fbox, boxrule=1pt, pad at break*=1mm,colback=cellbackground, colframe=cellborder]
\prompt{In}{incolor}{4}{\boxspacing}
\begin{Verbatim}[commandchars=\\\{\}]
\PY{n}{train\PYZus{}images\PYZus{}reshaped} \PY{o}{=} \PY{n}{train\PYZus{}images}\PY{o}{.}\PY{n}{reshape}\PY{p}{(}\PY{p}{(}\PY{n}{train\PYZus{}images}\PY{o}{.}\PY{n}{shape}\PY{p}{[}\PY{l+m+mi}{0}\PY{p}{]}\PY{p}{,} \PY{l+m+mi}{28}\PY{o}{*}\PY{l+m+mi}{28}\PY{p}{)}\PY{p}{)}

\PY{n+nb}{print}\PY{p}{(}\PY{l+s+sa}{f}\PY{l+s+s1}{\PYZsq{}}\PY{l+s+s1}{Train images before reshape: }\PY{l+s+si}{\PYZob{}train\PYZus{}images.shape\PYZcb{}}\PY{l+s+s1}{\PYZsq{}}\PY{p}{)}
\PY{n+nb}{print}\PY{p}{(}\PY{l+s+sa}{f}\PY{l+s+s1}{\PYZsq{}}\PY{l+s+s1}{Train images after  reshape: }\PY{l+s+si}{\PYZob{}train\PYZus{}images\PYZus{}reshaped.shape\PYZcb{}}\PY{l+s+s1}{\PYZsq{}}\PY{p}{)}

\PY{k}{assert} \PY{n+nb}{all}\PY{p}{(}\PY{n}{train\PYZus{}images}\PY{p}{[}\PY{l+m+mi}{0}\PY{p}{,} \PY{l+m+mi}{20}\PY{p}{]} \PY{o}{==} \PY{n}{train\PYZus{}images\PYZus{}reshaped}\PY{p}{[}\PY{l+m+mi}{0}\PY{p}{,} \PY{p}{(}\PY{l+m+mi}{20}\PY{p}{)}\PY{o}{*}\PY{l+m+mi}{28}\PY{p}{:}\PY{p}{(}\PY{l+m+mi}{20}\PY{o}{+}\PY{l+m+mi}{1}\PY{p}{)}\PY{o}{*}\PY{l+m+mi}{28}\PY{p}{]}\PY{p}{)}\PY{p}{,} \PY{l+s+s2}{\PYZdq{}}\PY{l+s+s2}{train reshaping is wrong}\PY{l+s+s2}{\PYZdq{}}
\end{Verbatim}
\end{tcolorbox}

    \begin{Verbatim}[commandchars=\\\{\}]
Train images before reshape: (60000, 28, 28)
Train images after  reshape: (60000, 784)
    \end{Verbatim}
\pagebreak

    We will use the functions defined below to answer the questions:

    \begin{tcolorbox}[breakable, size=fbox, boxrule=1pt, pad at break*=1mm,colback=cellbackground, colframe=cellborder]
\prompt{In}{incolor}{5}{\boxspacing}
\begin{Verbatim}[commandchars=\\\{\}]
\PY{k}{def} \PY{n+nf}{compute\PYZus{}pca\PYZus{}eig\PYZus{}vec}\PY{p}{(}\PY{n}{X}\PY{p}{,} \PY{n}{num\PYZus{}of\PYZus{}pca\PYZus{}components}\PY{p}{)}\PY{p}{:}
    \PY{n}{covar\PYZus{}matrix} \PY{o}{=} \PY{n}{np}\PY{o}{.}\PY{n}{matmul}\PY{p}{(}\PY{n}{X}\PY{o}{.}\PY{n}{T} \PY{p}{,} \PY{n}{X}\PY{p}{)}
    \PY{n}{\PYZus{}}\PY{p}{,} \PY{n}{vectors} \PY{o}{=} \PY{n}{eigh}\PY{p}{(}\PY{n}{covar\PYZus{}matrix}\PY{p}{)}
    \PY{n}{descending\PYZus{}vectors} \PY{o}{=} \PY{n}{vectors}\PY{p}{[}\PY{p}{:}\PY{p}{,}\PY{p}{:}\PY{p}{:}\PY{o}{\PYZhy{}}\PY{l+m+mi}{1}\PY{p}{]} 
    \PY{k}{return} \PY{n}{descending\PYZus{}vectors}\PY{p}{[}\PY{p}{:}\PY{p}{,} \PY{l+m+mi}{0}\PY{p}{:}\PY{n}{num\PYZus{}of\PYZus{}pca\PYZus{}components}\PY{p}{]}


\PY{k}{def} \PY{n+nf}{compute\PYZus{}x\PYZus{}reduced} \PY{p}{(}\PY{n}{X}\PY{p}{,} \PY{n}{eig\PYZus{}vectors}\PY{p}{)}\PY{p}{:}
    \PY{k}{return} \PY{n}{np}\PY{o}{.}\PY{n}{matmul}\PY{p}{(}\PY{n}{X}\PY{p}{,} \PY{n}{eig\PYZus{}vectors}\PY{p}{)}


\PY{k}{def} \PY{n+nf}{compute\PYZus{}x\PYZus{}reconstruct} \PY{p}{(}\PY{n}{X\PYZus{}reduced}\PY{p}{,} \PY{n}{eig\PYZus{}vectors}\PY{p}{)}\PY{p}{:}
    \PY{k}{return} \PY{n}{np}\PY{o}{.}\PY{n}{matmul}\PY{p}{(}\PY{n}{X\PYZus{}reduced}\PY{p}{,} \PY{n}{eig\PYZus{}vectors}\PY{o}{.}\PY{n}{T}\PY{p}{)}
    
    
\PY{k}{def} \PY{n+nf}{dispaly\PYZus{}eigs}\PY{p}{(}\PY{n}{eigs}\PY{p}{)}\PY{p}{:}
    \PY{n}{fig}\PY{p}{,} \PY{n}{ax} \PY{o}{=} \PY{n}{plt}\PY{o}{.}\PY{n}{subplots}\PY{p}{(}\PY{n}{eigs}\PY{o}{.}\PY{n}{shape}\PY{p}{[}\PY{l+m+mi}{0}\PY{p}{]}\PY{o}{/}\PY{o}{/}\PY{l+m+mi}{5}\PY{p}{,} \PY{l+m+mi}{5}\PY{p}{,} \PY{n}{figsize}\PY{o}{=}\PY{p}{(}\PY{l+m+mi}{20}\PY{p}{,} \PY{l+m+mi}{20}\PY{p}{)}\PY{p}{)}
    \PY{k}{for} \PY{n}{i} \PY{o+ow}{in} \PY{n+nb}{range}\PY{p}{(}\PY{n}{eigs}\PY{o}{.}\PY{n}{shape}\PY{p}{[}\PY{l+m+mi}{0}\PY{p}{]}\PY{o}{/}\PY{o}{/}\PY{l+m+mi}{5}\PY{p}{)}\PY{p}{:}
        \PY{k}{for} \PY{n}{j} \PY{o+ow}{in} \PY{n+nb}{range}\PY{p}{(}\PY{l+m+mi}{5}\PY{p}{)}\PY{p}{:}
            \PY{n}{ax}\PY{p}{[}\PY{n}{i}\PY{p}{]}\PY{p}{[}\PY{n}{j}\PY{p}{]}\PY{o}{.}\PY{n}{imshow}\PY{p}{(}\PY{n}{eigs}\PY{p}{[}\PY{l+m+mi}{5} \PY{o}{*} \PY{n}{i} \PY{o}{+} \PY{n}{j}\PY{p}{]}\PY{o}{.}\PY{n}{reshape}\PY{p}{(}\PY{l+m+mi}{28}\PY{p}{,} \PY{l+m+mi}{28}\PY{p}{)}\PY{p}{,} \PY{n}{cmap}\PY{o}{=}\PY{l+s+s1}{\PYZsq{}}\PY{l+s+s1}{Greys}\PY{l+s+s1}{\PYZsq{}}\PY{p}{)}
            \PY{n}{ax}\PY{p}{[}\PY{n}{i}\PY{p}{]}\PY{p}{[}\PY{n}{j}\PY{p}{]}\PY{o}{.}\PY{n}{xaxis}\PY{o}{.}\PY{n}{set\PYZus{}visible}\PY{p}{(}\PY{k+kc}{False}\PY{p}{)}
            \PY{n}{ax}\PY{p}{[}\PY{n}{i}\PY{p}{]}\PY{p}{[}\PY{n}{j}\PY{p}{]}\PY{o}{.}\PY{n}{yaxis}\PY{o}{.}\PY{n}{set\PYZus{}visible}\PY{p}{(}\PY{k+kc}{False}\PY{p}{)}
    \PY{n}{plt}\PY{o}{.}\PY{n}{show}\PY{p}{(}\PY{p}{)}
    

\PY{k}{def} \PY{n+nf}{dispaly\PYZus{}reconstruct}\PY{p}{(}\PY{n}{x\PYZus{}reconstruct}\PY{p}{)}\PY{p}{:}
    \PY{n}{plt}\PY{o}{.}\PY{n}{imshow}\PY{p}{(}\PY{n}{x\PYZus{}reconstruct}\PY{o}{.}\PY{n}{reshape}\PY{p}{(}\PY{l+m+mi}{28}\PY{p}{,} \PY{l+m+mi}{28}\PY{p}{)}\PY{p}{,} \PY{n}{cmap}\PY{o}{=}\PY{l+s+s1}{\PYZsq{}}\PY{l+s+s1}{Greys}\PY{l+s+s1}{\PYZsq{}}\PY{p}{)}
    \PY{n}{plt}\PY{o}{.}\PY{n}{axis}\PY{p}{(}\PY{l+s+s1}{\PYZsq{}}\PY{l+s+s1}{off}\PY{l+s+s1}{\PYZsq{}}\PY{p}{)}
    \PY{n}{plt}\PY{o}{.}\PY{n}{show}\PY{p}{(}\PY{p}{)}
    
    
\PY{k}{def} \PY{n+nf}{different\PYZus{}pcas}\PY{p}{(}\PY{n}{X}\PY{p}{,} \PY{n}{num\PYZus{}of\PYZus{}eig}\PY{p}{)}\PY{p}{:}
    \PY{n}{scal} \PY{o}{=} \PY{n}{StandardScaler}\PY{p}{(}\PY{p}{)}
    \PY{n}{X\PYZus{}standard} \PY{o}{=} \PY{n}{scal}\PY{o}{.}\PY{n}{fit\PYZus{}transform}\PY{p}{(}\PY{n}{X}\PY{p}{)}
    \PY{n}{eig\PYZus{}vecs} \PY{o}{=} \PY{n}{compute\PYZus{}pca\PYZus{}eig\PYZus{}vec}\PY{p}{(}\PY{n}{X\PYZus{}standard}\PY{p}{,} \PY{n}{num\PYZus{}of\PYZus{}eig}\PY{p}{)}
    \PY{n+nb}{print}\PY{p}{(}\PY{l+s+s1}{\PYZsq{}}\PY{l+s+s1}{The first 10 eigen\PYZhy{}vectors:}\PY{l+s+s1}{\PYZsq{}}\PY{p}{)}
    \PY{n}{dispaly\PYZus{}eigs}\PY{p}{(}\PY{n}{eig\PYZus{}vecs}\PY{o}{.}\PY{n}{T}\PY{p}{)}
    
    \PY{n}{X\PYZus{}reduced\PYZus{}2} \PY{o}{=} \PY{n}{compute\PYZus{}x\PYZus{}reduced}\PY{p}{(}\PY{n}{X\PYZus{}standard}\PY{p}{[}\PY{l+m+mi}{0}\PY{p}{:}\PY{l+m+mi}{1}\PY{p}{]}\PY{p}{,} \PY{n}{eig\PYZus{}vecs}\PY{p}{[}\PY{p}{:}\PY{p}{,} \PY{l+m+mi}{0}\PY{p}{:}\PY{l+m+mi}{2}\PY{p}{]}\PY{p}{)}
    \PY{n}{X\PYZus{}reduced\PYZus{}5} \PY{o}{=} \PY{n}{compute\PYZus{}x\PYZus{}reduced}\PY{p}{(}\PY{n}{X\PYZus{}standard}\PY{p}{[}\PY{l+m+mi}{0}\PY{p}{:}\PY{l+m+mi}{1}\PY{p}{]}\PY{p}{,} \PY{n}{eig\PYZus{}vecs}\PY{p}{[}\PY{p}{:}\PY{p}{,} \PY{l+m+mi}{0}\PY{p}{:}\PY{l+m+mi}{5}\PY{p}{]}\PY{p}{)}
    \PY{n}{X\PYZus{}reduced\PYZus{}10} \PY{o}{=} \PY{n}{compute\PYZus{}x\PYZus{}reduced}\PY{p}{(}\PY{n}{X\PYZus{}standard}\PY{p}{[}\PY{l+m+mi}{0}\PY{p}{:}\PY{l+m+mi}{1}\PY{p}{]}\PY{p}{,} \PY{n}{eig\PYZus{}vecs}\PY{p}{[}\PY{p}{:}\PY{p}{,} \PY{l+m+mi}{0}\PY{p}{:}\PY{l+m+mi}{10}\PY{p}{]}\PY{p}{)}
    
    \PY{n}{X\PYZus{}reconstruct\PYZus{}2} \PY{o}{=} \PY{n}{compute\PYZus{}x\PYZus{}reconstruct}\PY{p}{(}\PY{n}{X\PYZus{}reduced\PYZus{}2}\PY{p}{,} \PY{n}{eig\PYZus{}vecs}\PY{p}{[}\PY{p}{:}\PY{p}{,} \PY{l+m+mi}{0}\PY{p}{:}\PY{l+m+mi}{2}\PY{p}{]}\PY{p}{)}
    \PY{n}{X\PYZus{}reconstruct\PYZus{}5} \PY{o}{=} \PY{n}{compute\PYZus{}x\PYZus{}reconstruct}\PY{p}{(}\PY{n}{X\PYZus{}reduced\PYZus{}5}\PY{p}{,} \PY{n}{eig\PYZus{}vecs}\PY{p}{[}\PY{p}{:}\PY{p}{,} \PY{l+m+mi}{0}\PY{p}{:}\PY{l+m+mi}{5}\PY{p}{]}\PY{p}{)}
    \PY{n}{X\PYZus{}reconstruct\PYZus{}10} \PY{o}{=} \PY{n}{compute\PYZus{}x\PYZus{}reconstruct}\PY{p}{(}\PY{n}{X\PYZus{}reduced\PYZus{}10}\PY{p}{,} \PY{n}{eig\PYZus{}vecs}\PY{p}{[}\PY{p}{:}\PY{p}{,} \PY{l+m+mi}{0}\PY{p}{:}\PY{l+m+mi}{10}\PY{p}{]}\PY{p}{)}
    
    \PY{n+nb}{print}\PY{p}{(}\PY{l+s+s1}{\PYZsq{}}\PY{l+s+s1}{Reconstruction of a sample from the first 2 eigen\PYZhy{}vectors:}\PY{l+s+s1}{\PYZsq{}}\PY{p}{)}
    \PY{n}{dispaly\PYZus{}reconstruct}\PY{p}{(}\PY{n}{scal}\PY{o}{.}\PY{n}{inverse\PYZus{}transform}\PY{p}{(}\PY{n}{X\PYZus{}reconstruct\PYZus{}2}\PY{p}{)}\PY{p}{)}
    \PY{n+nb}{print}\PY{p}{(}\PY{l+s+s1}{\PYZsq{}}\PY{l+s+s1}{Reconstruction of a sample from the first 5 eigen\PYZhy{}vectors:}\PY{l+s+s1}{\PYZsq{}}\PY{p}{)}
    \PY{n}{dispaly\PYZus{}reconstruct}\PY{p}{(}\PY{n}{scal}\PY{o}{.}\PY{n}{inverse\PYZus{}transform}\PY{p}{(}\PY{n}{X\PYZus{}reconstruct\PYZus{}5}\PY{p}{)}\PY{p}{)}
    \PY{n+nb}{print}\PY{p}{(}\PY{l+s+s1}{\PYZsq{}}\PY{l+s+s1}{Reconstruction of a sample from the first 10 eigen\PYZhy{}vectors:}\PY{l+s+s1}{\PYZsq{}}\PY{p}{)}
    \PY{n}{dispaly\PYZus{}reconstruct}\PY{p}{(}\PY{n}{scal}\PY{o}{.}\PY{n}{inverse\PYZus{}transform}\PY{p}{(}\PY{n}{X\PYZus{}reconstruct\PYZus{}10}\PY{p}{)}\PY{p}{)}
    
    \PY{k}{return} \PY{n}{X\PYZus{}reconstruct\PYZus{}2}\PY{p}{,} \PY{n}{X\PYZus{}reconstruct\PYZus{}5}\PY{p}{,} \PY{n}{X\PYZus{}reconstruct\PYZus{}10} 


\PY{k}{def} \PY{n+nf}{different\PYZus{}pcas\PYZus{}Q2}\PY{p}{(}\PY{n}{X}\PY{p}{,} \PY{n}{num\PYZus{}of\PYZus{}eig}\PY{p}{)}\PY{p}{:}
    \PY{n}{scal} \PY{o}{=} \PY{n}{StandardScaler}\PY{p}{(}\PY{p}{)}
    \PY{n}{X\PYZus{}standard} \PY{o}{=} \PY{n}{scal}\PY{o}{.}\PY{n}{fit\PYZus{}transform}\PY{p}{(}\PY{n}{X}\PY{p}{)}
    \PY{n}{eig\PYZus{}vecs} \PY{o}{=} \PY{n}{compute\PYZus{}pca\PYZus{}eig\PYZus{}vec}\PY{p}{(}\PY{n}{X\PYZus{}standard}\PY{p}{,} \PY{n}{num\PYZus{}of\PYZus{}eig}\PY{p}{)}
    \PY{n+nb}{print}\PY{p}{(}\PY{l+s+s1}{\PYZsq{}}\PY{l+s+s1}{The first 20 eigen\PYZhy{}vectors:}\PY{l+s+s1}{\PYZsq{}}\PY{p}{)}
    \PY{n}{dispaly\PYZus{}eigs}\PY{p}{(}\PY{n}{eig\PYZus{}vecs}\PY{o}{.}\PY{n}{T}\PY{p}{)}
    
    \PY{n}{digit\PYZus{}array} \PY{o}{=} \PY{n}{np}\PY{o}{.}\PY{n}{zeros}\PY{p}{(}\PY{p}{(}\PY{l+m+mi}{10}\PY{p}{,} \PY{l+m+mi}{784}\PY{p}{)}\PY{p}{)}
    \PY{k}{for} \PY{n}{i} \PY{o+ow}{in} \PY{n+nb}{range}\PY{p}{(}\PY{l+m+mi}{10}\PY{p}{)}\PY{p}{:}
        \PY{n}{digit\PYZus{}array}\PY{p}{[}\PY{n}{i}\PY{p}{,} \PY{p}{:}\PY{p}{]} \PY{o}{=} \PY{n}{X\PYZus{}standard}\PY{p}{[}\PY{n}{train\PYZus{}labels} \PY{o}{==} \PY{n}{i}\PY{p}{,} \PY{p}{:}\PY{p}{]}\PY{p}{[}\PY{l+m+mi}{1}\PY{p}{]}
    
    \PY{n+nb}{print}\PY{p}{(}\PY{l+s+s1}{\PYZsq{}}\PY{l+s+s1}{10 samples have been selected of every digit group:}\PY{l+s+s1}{\PYZsq{}}\PY{p}{)}
    \PY{n}{dispaly\PYZus{}eigs}\PY{p}{(}\PY{n}{scal}\PY{o}{.}\PY{n}{inverse\PYZus{}transform}\PY{p}{(}\PY{n}{digit\PYZus{}array}\PY{p}{)}\PY{p}{)}

    \PY{n}{X\PYZus{}reduced\PYZus{}2} \PY{o}{=} \PY{n}{compute\PYZus{}x\PYZus{}reduced}\PY{p}{(}\PY{n}{digit\PYZus{}array}\PY{p}{,} \PY{n}{eig\PYZus{}vecs}\PY{p}{[}\PY{p}{:}\PY{p}{,} \PY{l+m+mi}{0}\PY{p}{:}\PY{l+m+mi}{2}\PY{p}{]}\PY{p}{)}
    \PY{n}{X\PYZus{}reduced\PYZus{}5} \PY{o}{=} \PY{n}{compute\PYZus{}x\PYZus{}reduced}\PY{p}{(}\PY{n}{digit\PYZus{}array}\PY{p}{,} \PY{n}{eig\PYZus{}vecs}\PY{p}{[}\PY{p}{:}\PY{p}{,} \PY{l+m+mi}{0}\PY{p}{:}\PY{l+m+mi}{5}\PY{p}{]}\PY{p}{)}
    \PY{n}{X\PYZus{}reduced\PYZus{}10} \PY{o}{=} \PY{n}{compute\PYZus{}x\PYZus{}reduced}\PY{p}{(}\PY{n}{digit\PYZus{}array}\PY{p}{,} \PY{n}{eig\PYZus{}vecs}\PY{p}{[}\PY{p}{:}\PY{p}{,} \PY{l+m+mi}{0}\PY{p}{:}\PY{l+m+mi}{10}\PY{p}{]}\PY{p}{)}
    
    
    \PY{n}{X\PYZus{}reconstruct\PYZus{}2} \PY{o}{=} \PY{n}{compute\PYZus{}x\PYZus{}reconstruct}\PY{p}{(}\PY{n}{X\PYZus{}reduced\PYZus{}2}\PY{p}{,} \PY{n}{eig\PYZus{}vecs}\PY{p}{[}\PY{p}{:}\PY{p}{,} \PY{l+m+mi}{0}\PY{p}{:}\PY{l+m+mi}{2}\PY{p}{]}\PY{p}{)}
    \PY{n}{X\PYZus{}reconstruct\PYZus{}5} \PY{o}{=} \PY{n}{compute\PYZus{}x\PYZus{}reconstruct}\PY{p}{(}\PY{n}{X\PYZus{}reduced\PYZus{}5}\PY{p}{,} \PY{n}{eig\PYZus{}vecs}\PY{p}{[}\PY{p}{:}\PY{p}{,} \PY{l+m+mi}{0}\PY{p}{:}\PY{l+m+mi}{5}\PY{p}{]}\PY{p}{)}
    \PY{n}{X\PYZus{}reconstruct\PYZus{}10} \PY{o}{=} \PY{n}{compute\PYZus{}x\PYZus{}reconstruct}\PY{p}{(}\PY{n}{X\PYZus{}reduced\PYZus{}10}\PY{p}{,} \PY{n}{eig\PYZus{}vecs}\PY{p}{[}\PY{p}{:}\PY{p}{,} \PY{l+m+mi}{0}\PY{p}{:}\PY{l+m+mi}{10}\PY{p}{]}\PY{p}{)}
        
    
    \PY{n+nb}{print}\PY{p}{(}\PY{l+s+s1}{\PYZsq{}}\PY{l+s+s1}{Reconstruction of samples from the first 2 eigen\PYZhy{}vectors:}\PY{l+s+s1}{\PYZsq{}}\PY{p}{)}
    \PY{n}{dispaly\PYZus{}eigs}\PY{p}{(}\PY{n}{scal}\PY{o}{.}\PY{n}{inverse\PYZus{}transform}\PY{p}{(}\PY{n}{X\PYZus{}reconstruct\PYZus{}2}\PY{p}{)}\PY{p}{)}
    \PY{n+nb}{print}\PY{p}{(}\PY{l+s+s1}{\PYZsq{}}\PY{l+s+s1}{Reconstruction of samples from the first 5 eigen\PYZhy{}vectors:}\PY{l+s+s1}{\PYZsq{}}\PY{p}{)}
    \PY{n}{dispaly\PYZus{}eigs}\PY{p}{(}\PY{n}{scal}\PY{o}{.}\PY{n}{inverse\PYZus{}transform}\PY{p}{(}\PY{n}{X\PYZus{}reconstruct\PYZus{}5}\PY{p}{)}\PY{p}{)}
    \PY{n+nb}{print}\PY{p}{(}\PY{l+s+s1}{\PYZsq{}}\PY{l+s+s1}{Reconstruction of samples from the first 10 eigen\PYZhy{}vectors:}\PY{l+s+s1}{\PYZsq{}}\PY{p}{)}
    \PY{n}{dispaly\PYZus{}eigs}\PY{p}{(}\PY{n}{scal}\PY{o}{.}\PY{n}{inverse\PYZus{}transform}\PY{p}{(}\PY{n}{X\PYZus{}reconstruct\PYZus{}10}\PY{p}{)}\PY{p}{)}
    
    \PY{k}{return} \PY{n}{X\PYZus{}reconstruct\PYZus{}2}\PY{p}{,} \PY{n}{X\PYZus{}reconstruct\PYZus{}5}\PY{p}{,} \PY{n}{X\PYZus{}reconstruct\PYZus{}10} 
\end{Verbatim}
\end{tcolorbox}
\pagebreak
    \hypertarget{question-1}{%
\section{Question 1}\label{question-1}}

\begin{center}\rule{0.5\linewidth}{0.5pt}\end{center}

    In the first question we only need to know the first 10 eigen-vectors:

    \begin{tcolorbox}[breakable, size=fbox, boxrule=1pt, pad at break*=1mm,colback=cellbackground, colframe=cellborder]
\prompt{In}{incolor}{6}{\boxspacing}
\begin{Verbatim}[commandchars=\\\{\}]
\PY{n}{num\PYZus{}of\PYZus{}eig} \PY{o}{=} \PY{l+m+mi}{10}
\end{Verbatim}
\end{tcolorbox}

    In the 10 following sectons, for group of unique digits, we will show
the first 10 eigenvectors as images, and will reconstruct 3 versions of
one sample of the group as listed below: - Reconstruction from the first
2 eigen-vectors - Reconstruction from the first 5 eigen-vectors -
Reconstruction from the first 10 eigen-vectors

\pagebreak
    \hypertarget{digit-0}{%
\subsection{Digit 0}\label{digit-0}}

    \begin{tcolorbox}[breakable, size=fbox, boxrule=1pt, pad at break*=1mm,colback=cellbackground, colframe=cellborder]
\prompt{In}{incolor}{7}{\boxspacing}
\begin{Verbatim}[commandchars=\\\{\}]
\PY{n}{X} \PY{o}{=} \PY{n}{train\PYZus{}images\PYZus{}reshaped}\PY{p}{[}\PY{n}{train\PYZus{}labels}\PY{o}{==}\PY{l+m+mi}{0}\PY{p}{,} \PY{p}{:}\PY{p}{]}
\PY{n}{X\PYZus{}reconstruct\PYZus{}2}\PY{p}{,} \PY{n}{X\PYZus{}reconstruct\PYZus{}5}\PY{p}{,} \PY{n}{X\PYZus{}reconstruct\PYZus{}10} \PY{o}{=} \PY{n}{different\PYZus{}pcas}\PY{p}{(}\PY{n}{X}\PY{p}{,} \PY{n}{num\PYZus{}of\PYZus{}eig}\PY{p}{)}
\end{Verbatim}
\end{tcolorbox}

    \begin{Verbatim}[commandchars=\\\{\}]
The first 10 eigen-vectors:
    \end{Verbatim}

    \begin{center}
    \adjustimage{max size={0.7\linewidth}{0.7\paperheight}}{output_15_1.png}
    \end{center}
    { \hspace*{\fill} \\}
    
    \begin{Verbatim}[commandchars=\\\{\}]
Reconstruction of a sample from the first 2 eigen-vectors:
    \end{Verbatim}

    \begin{center}
    \adjustimage{max size={0.7\linewidth}{0.7\paperheight}}{output_15_3.png}
    \end{center}
    { \hspace*{\fill} \\}
    
    \begin{Verbatim}[commandchars=\\\{\}]
Reconstruction of a sample from the first 5 eigen-vectors:
    \end{Verbatim}

    \begin{center}
    \adjustimage{max size={0.7\linewidth}{0.7\paperheight}}{output_15_5.png}
    \end{center}
    { \hspace*{\fill} \\}
    
    \begin{Verbatim}[commandchars=\\\{\}]
Reconstruction of a sample from the first 10 eigen-vectors:
    \end{Verbatim}

    \begin{center}
    \adjustimage{max size={0.7\linewidth}{0.7\paperheight}}{output_15_7.png}
    \end{center}
    { \hspace*{\fill} \\}
    \pagebreak
    
    \hypertarget{digit-1}{%
\subsection{Digit 1}\label{digit-1}}

    \begin{tcolorbox}[breakable, size=fbox, boxrule=1pt, pad at break*=1mm,colback=cellbackground, colframe=cellborder]
\prompt{In}{incolor}{8}{\boxspacing}
\begin{Verbatim}[commandchars=\\\{\}]
\PY{n}{X} \PY{o}{=} \PY{n}{train\PYZus{}images\PYZus{}reshaped}\PY{p}{[}\PY{n}{train\PYZus{}labels}\PY{o}{==}\PY{l+m+mi}{1}\PY{p}{,} \PY{p}{:}\PY{p}{]}
\PY{n}{X\PYZus{}reconstruct\PYZus{}2}\PY{p}{,} \PY{n}{X\PYZus{}reconstruct\PYZus{}5}\PY{p}{,} \PY{n}{X\PYZus{}reconstruct\PYZus{}10} \PY{o}{=} \PY{n}{different\PYZus{}pcas}\PY{p}{(}\PY{n}{X}\PY{p}{,} \PY{n}{num\PYZus{}of\PYZus{}eig}\PY{p}{)}
\end{Verbatim}
\end{tcolorbox}

    \begin{Verbatim}[commandchars=\\\{\}]
The first 10 eigen-vectors:
    \end{Verbatim}

    \begin{center}
    \adjustimage{max size={0.7\linewidth}{0.7\paperheight}}{output_17_1.png}
    \end{center}
    { \hspace*{\fill} \\}
    
    \begin{Verbatim}[commandchars=\\\{\}]
Reconstruction of a sample from the first 2 eigen-vectors:
    \end{Verbatim}

    \begin{center}
    \adjustimage{max size={0.7\linewidth}{0.7\paperheight}}{output_17_3.png}
    \end{center}
    { \hspace*{\fill} \\}
    
    \begin{Verbatim}[commandchars=\\\{\}]
Reconstruction of a sample from the first 5 eigen-vectors:
    \end{Verbatim}

    \begin{center}
    \adjustimage{max size={0.7\linewidth}{0.7\paperheight}}{output_17_5.png}
    \end{center}
    { \hspace*{\fill} \\}
    
    \begin{Verbatim}[commandchars=\\\{\}]
Reconstruction of a sample from the first 10 eigen-vectors:
    \end{Verbatim}

    \begin{center}
    \adjustimage{max size={0.7\linewidth}{0.7\paperheight}}{output_17_7.png}
    \end{center}
    { \hspace*{\fill} \\}
    \pagebreak
    
    \hypertarget{digit-2}{%
\subsection{Digit 2}\label{digit-2}}

    \begin{tcolorbox}[breakable, size=fbox, boxrule=1pt, pad at break*=1mm,colback=cellbackground, colframe=cellborder]
\prompt{In}{incolor}{9}{\boxspacing}
\begin{Verbatim}[commandchars=\\\{\}]
\PY{n}{X} \PY{o}{=} \PY{n}{train\PYZus{}images\PYZus{}reshaped}\PY{p}{[}\PY{n}{train\PYZus{}labels}\PY{o}{==}\PY{l+m+mi}{2}\PY{p}{,} \PY{p}{:}\PY{p}{]}
\PY{n}{X\PYZus{}reconstruct\PYZus{}2}\PY{p}{,} \PY{n}{X\PYZus{}reconstruct\PYZus{}5}\PY{p}{,} \PY{n}{X\PYZus{}reconstruct\PYZus{}10} \PY{o}{=} \PY{n}{different\PYZus{}pcas}\PY{p}{(}\PY{n}{X}\PY{p}{,} \PY{n}{num\PYZus{}of\PYZus{}eig}\PY{p}{)}
\end{Verbatim}
\end{tcolorbox}

    \begin{Verbatim}[commandchars=\\\{\}]
The first 10 eigen-vectors:
    \end{Verbatim}

    \begin{center}
    \adjustimage{max size={0.7\linewidth}{0.7\paperheight}}{output_19_1.png}
    \end{center}
    { \hspace*{\fill} \\}
    
    \begin{Verbatim}[commandchars=\\\{\}]
Reconstruction of a sample from the first 2 eigen-vectors:
    \end{Verbatim}

    \begin{center}
    \adjustimage{max size={0.7\linewidth}{0.7\paperheight}}{output_19_3.png}
    \end{center}
    { \hspace*{\fill} \\}
    
    \begin{Verbatim}[commandchars=\\\{\}]
Reconstruction of a sample from the first 5 eigen-vectors:
    \end{Verbatim}

    \begin{center}
    \adjustimage{max size={0.7\linewidth}{0.7\paperheight}}{output_19_5.png}
    \end{center}
    { \hspace*{\fill} \\}
    
    \begin{Verbatim}[commandchars=\\\{\}]
Reconstruction of a sample from the first 10 eigen-vectors:
    \end{Verbatim}

    \begin{center}
    \adjustimage{max size={0.7\linewidth}{0.7\paperheight}}{output_19_7.png}
    \end{center}
    { \hspace*{\fill} \\}
    \pagebreak
    
    \hypertarget{digit-3}{%
\subsection{Digit 3}\label{digit-3}}

    \begin{tcolorbox}[breakable, size=fbox, boxrule=1pt, pad at break*=1mm,colback=cellbackground, colframe=cellborder]
\prompt{In}{incolor}{10}{\boxspacing}
\begin{Verbatim}[commandchars=\\\{\}]
\PY{n}{X} \PY{o}{=} \PY{n}{train\PYZus{}images\PYZus{}reshaped}\PY{p}{[}\PY{n}{train\PYZus{}labels}\PY{o}{==}\PY{l+m+mi}{3}\PY{p}{,} \PY{p}{:}\PY{p}{]}
\PY{n}{X\PYZus{}reconstruct\PYZus{}2}\PY{p}{,} \PY{n}{X\PYZus{}reconstruct\PYZus{}5}\PY{p}{,} \PY{n}{X\PYZus{}reconstruct\PYZus{}10} \PY{o}{=} \PY{n}{different\PYZus{}pcas}\PY{p}{(}\PY{n}{X}\PY{p}{,} \PY{n}{num\PYZus{}of\PYZus{}eig}\PY{p}{)}
\end{Verbatim}
\end{tcolorbox}

    \begin{Verbatim}[commandchars=\\\{\}]
The first 10 eigen-vectors:
    \end{Verbatim}

    \begin{center}
    \adjustimage{max size={0.7\linewidth}{0.7\paperheight}}{output_21_1.png}
    \end{center}
    { \hspace*{\fill} \\}
    
    \begin{Verbatim}[commandchars=\\\{\}]
Reconstruction of a sample from the first 2 eigen-vectors:
    \end{Verbatim}

    \begin{center}
    \adjustimage{max size={0.7\linewidth}{0.7\paperheight}}{output_21_3.png}
    \end{center}
    { \hspace*{\fill} \\}
    
    \begin{Verbatim}[commandchars=\\\{\}]
Reconstruction of a sample from the first 5 eigen-vectors:
    \end{Verbatim}

    \begin{center}
    \adjustimage{max size={0.7\linewidth}{0.7\paperheight}}{output_21_5.png}
    \end{center}
    { \hspace*{\fill} \\}
    
    \begin{Verbatim}[commandchars=\\\{\}]
Reconstruction of a sample from the first 10 eigen-vectors:
    \end{Verbatim}

    \begin{center}
    \adjustimage{max size={0.7\linewidth}{0.7\paperheight}}{output_21_7.png}
    \end{center}
    { \hspace*{\fill} \\}
    \pagebreak
    
    \hypertarget{digit-4}{%
\subsection{Digit 4}\label{digit-4}}

    \begin{tcolorbox}[breakable, size=fbox, boxrule=1pt, pad at break*=1mm,colback=cellbackground, colframe=cellborder]
\prompt{In}{incolor}{11}{\boxspacing}
\begin{Verbatim}[commandchars=\\\{\}]
\PY{n}{X} \PY{o}{=} \PY{n}{train\PYZus{}images\PYZus{}reshaped}\PY{p}{[}\PY{n}{train\PYZus{}labels}\PY{o}{==}\PY{l+m+mi}{4}\PY{p}{,} \PY{p}{:}\PY{p}{]}
\PY{n}{X\PYZus{}reconstruct\PYZus{}2}\PY{p}{,} \PY{n}{X\PYZus{}reconstruct\PYZus{}5}\PY{p}{,} \PY{n}{X\PYZus{}reconstruct\PYZus{}10} \PY{o}{=} \PY{n}{different\PYZus{}pcas}\PY{p}{(}\PY{n}{X}\PY{p}{,} \PY{n}{num\PYZus{}of\PYZus{}eig}\PY{p}{)}
\end{Verbatim}
\end{tcolorbox}

    \begin{Verbatim}[commandchars=\\\{\}]
The first 10 eigen-vectors:
    \end{Verbatim}

    \begin{center}
    \adjustimage{max size={0.7\linewidth}{0.7\paperheight}}{output_23_1.png}
    \end{center}
    { \hspace*{\fill} \\}
    
    \begin{Verbatim}[commandchars=\\\{\}]
Reconstruction of a sample from the first 2 eigen-vectors:
    \end{Verbatim}

    \begin{center}
    \adjustimage{max size={0.7\linewidth}{0.7\paperheight}}{output_23_3.png}
    \end{center}
    { \hspace*{\fill} \\}
    
    \begin{Verbatim}[commandchars=\\\{\}]
Reconstruction of a sample from the first 5 eigen-vectors:
    \end{Verbatim}

    \begin{center}
    \adjustimage{max size={0.7\linewidth}{0.7\paperheight}}{output_23_5.png}
    \end{center}
    { \hspace*{\fill} \\}
    
    \begin{Verbatim}[commandchars=\\\{\}]
Reconstruction of a sample from the first 10 eigen-vectors:
    \end{Verbatim}

    \begin{center}
    \adjustimage{max size={0.7\linewidth}{0.7\paperheight}}{output_23_7.png}
    \end{center}
    { \hspace*{\fill} \\}
    \pagebreak
    
    \hypertarget{digit-5}{%
\subsection{Digit 5}\label{digit-5}}

    \begin{tcolorbox}[breakable, size=fbox, boxrule=1pt, pad at break*=1mm,colback=cellbackground, colframe=cellborder]
\prompt{In}{incolor}{12}{\boxspacing}
\begin{Verbatim}[commandchars=\\\{\}]
\PY{n}{X} \PY{o}{=} \PY{n}{train\PYZus{}images\PYZus{}reshaped}\PY{p}{[}\PY{n}{train\PYZus{}labels}\PY{o}{==}\PY{l+m+mi}{5}\PY{p}{,} \PY{p}{:}\PY{p}{]}
\PY{n}{X\PYZus{}reconstruct\PYZus{}2}\PY{p}{,} \PY{n}{X\PYZus{}reconstruct\PYZus{}5}\PY{p}{,} \PY{n}{X\PYZus{}reconstruct\PYZus{}10} \PY{o}{=} \PY{n}{different\PYZus{}pcas}\PY{p}{(}\PY{n}{X}\PY{p}{,} \PY{n}{num\PYZus{}of\PYZus{}eig}\PY{p}{)}
\end{Verbatim}
\end{tcolorbox}

    \begin{Verbatim}[commandchars=\\\{\}]
The first 10 eigen-vectors:
    \end{Verbatim}

    \begin{center}
    \adjustimage{max size={0.7\linewidth}{0.7\paperheight}}{output_25_1.png}
    \end{center}
    { \hspace*{\fill} \\}
    
    \begin{Verbatim}[commandchars=\\\{\}]
Reconstruction of a sample from the first 2 eigen-vectors:
    \end{Verbatim}

    \begin{center}
    \adjustimage{max size={0.7\linewidth}{0.7\paperheight}}{output_25_3.png}
    \end{center}
    { \hspace*{\fill} \\}
    
    \begin{Verbatim}[commandchars=\\\{\}]
Reconstruction of a sample from the first 5 eigen-vectors:
    \end{Verbatim}

    \begin{center}
    \adjustimage{max size={0.7\linewidth}{0.7\paperheight}}{output_25_5.png}
    \end{center}
    { \hspace*{\fill} \\}
    
    \begin{Verbatim}[commandchars=\\\{\}]
Reconstruction of a sample from the first 10 eigen-vectors:
    \end{Verbatim}

    \begin{center}
    \adjustimage{max size={0.7\linewidth}{0.7\paperheight}}{output_25_7.png}
    \end{center}
    { \hspace*{\fill} \\}
    \pagebreak
    
    \hypertarget{digit-6}{%
\subsection{Digit 6}\label{digit-6}}

    \begin{tcolorbox}[breakable, size=fbox, boxrule=1pt, pad at break*=1mm,colback=cellbackground, colframe=cellborder]
\prompt{In}{incolor}{13}{\boxspacing}
\begin{Verbatim}[commandchars=\\\{\}]
\PY{n}{X} \PY{o}{=} \PY{n}{train\PYZus{}images\PYZus{}reshaped}\PY{p}{[}\PY{n}{train\PYZus{}labels}\PY{o}{==}\PY{l+m+mi}{6}\PY{p}{,} \PY{p}{:}\PY{p}{]}
\PY{n}{X\PYZus{}reconstruct\PYZus{}2}\PY{p}{,} \PY{n}{X\PYZus{}reconstruct\PYZus{}5}\PY{p}{,} \PY{n}{X\PYZus{}reconstruct\PYZus{}10} \PY{o}{=} \PY{n}{different\PYZus{}pcas}\PY{p}{(}\PY{n}{X}\PY{p}{,} \PY{n}{num\PYZus{}of\PYZus{}eig}\PY{p}{)}
\end{Verbatim}
\end{tcolorbox}

    \begin{Verbatim}[commandchars=\\\{\}]
The first 10 eigen-vectors:
    \end{Verbatim}

    \begin{center}
    \adjustimage{max size={0.7\linewidth}{0.7\paperheight}}{output_27_1.png}
    \end{center}
    { \hspace*{\fill} \\}
    
    \begin{Verbatim}[commandchars=\\\{\}]
Reconstruction of a sample from the first 2 eigen-vectors:
    \end{Verbatim}

    \begin{center}
    \adjustimage{max size={0.7\linewidth}{0.7\paperheight}}{output_27_3.png}
    \end{center}
    { \hspace*{\fill} \\}
    
    \begin{Verbatim}[commandchars=\\\{\}]
Reconstruction of a sample from the first 5 eigen-vectors:
    \end{Verbatim}

    \begin{center}
    \adjustimage{max size={0.7\linewidth}{0.7\paperheight}}{output_27_5.png}
    \end{center}
    { \hspace*{\fill} \\}
    
    \begin{Verbatim}[commandchars=\\\{\}]
Reconstruction of a sample from the first 10 eigen-vectors:
    \end{Verbatim}

    \begin{center}
    \adjustimage{max size={0.7\linewidth}{0.7\paperheight}}{output_27_7.png}
    \end{center}
    { \hspace*{\fill} \\}
    \pagebreak
    
    \hypertarget{digit-7}{%
\subsection{Digit 7}\label{digit-7}}

    \begin{tcolorbox}[breakable, size=fbox, boxrule=1pt, pad at break*=1mm,colback=cellbackground, colframe=cellborder]
\prompt{In}{incolor}{14}{\boxspacing}
\begin{Verbatim}[commandchars=\\\{\}]
\PY{n}{X} \PY{o}{=} \PY{n}{train\PYZus{}images\PYZus{}reshaped}\PY{p}{[}\PY{n}{train\PYZus{}labels}\PY{o}{==}\PY{l+m+mi}{7}\PY{p}{,} \PY{p}{:}\PY{p}{]}
\PY{n}{X\PYZus{}reconstruct\PYZus{}2}\PY{p}{,} \PY{n}{X\PYZus{}reconstruct\PYZus{}5}\PY{p}{,} \PY{n}{X\PYZus{}reconstruct\PYZus{}10} \PY{o}{=} \PY{n}{different\PYZus{}pcas}\PY{p}{(}\PY{n}{X}\PY{p}{,} \PY{n}{num\PYZus{}of\PYZus{}eig}\PY{p}{)}
\end{Verbatim}
\end{tcolorbox}

    \begin{Verbatim}[commandchars=\\\{\}]
The first 10 eigen-vectors:
    \end{Verbatim}

    \begin{center}
    \adjustimage{max size={0.7\linewidth}{0.7\paperheight}}{output_29_1.png}
    \end{center}
    { \hspace*{\fill} \\}
    
    \begin{Verbatim}[commandchars=\\\{\}]
Reconstruction of a sample from the first 2 eigen-vectors:
    \end{Verbatim}

    \begin{center}
    \adjustimage{max size={0.7\linewidth}{0.7\paperheight}}{output_29_3.png}
    \end{center}
    { \hspace*{\fill} \\}
    
    \begin{Verbatim}[commandchars=\\\{\}]
Reconstruction of a sample from the first 5 eigen-vectors:
    \end{Verbatim}

    \begin{center}
    \adjustimage{max size={0.7\linewidth}{0.7\paperheight}}{output_29_5.png}
    \end{center}
    { \hspace*{\fill} \\}
    
    \begin{Verbatim}[commandchars=\\\{\}]
Reconstruction of a sample from the first 10 eigen-vectors:
    \end{Verbatim}

    \begin{center}
    \adjustimage{max size={0.7\linewidth}{0.7\paperheight}}{output_29_7.png}
    \end{center}
    { \hspace*{\fill} \\}
    \pagebreak
    
    \hypertarget{digit-8}{%
\subsection{Digit 8}\label{digit-8}}

    \begin{tcolorbox}[breakable, size=fbox, boxrule=1pt, pad at break*=1mm,colback=cellbackground, colframe=cellborder]
\prompt{In}{incolor}{15}{\boxspacing}
\begin{Verbatim}[commandchars=\\\{\}]
\PY{n}{X} \PY{o}{=} \PY{n}{train\PYZus{}images\PYZus{}reshaped}\PY{p}{[}\PY{n}{train\PYZus{}labels}\PY{o}{==}\PY{l+m+mi}{8}\PY{p}{,} \PY{p}{:}\PY{p}{]}
\PY{n}{X\PYZus{}reconstruct\PYZus{}2}\PY{p}{,} \PY{n}{X\PYZus{}reconstruct\PYZus{}5}\PY{p}{,} \PY{n}{X\PYZus{}reconstruct\PYZus{}10} \PY{o}{=} \PY{n}{different\PYZus{}pcas}\PY{p}{(}\PY{n}{X}\PY{p}{,} \PY{n}{num\PYZus{}of\PYZus{}eig}\PY{p}{)}
\end{Verbatim}
\end{tcolorbox}

    \begin{Verbatim}[commandchars=\\\{\}]
The first 10 eigen-vectors:
    \end{Verbatim}

    \begin{center}
    \adjustimage{max size={0.7\linewidth}{0.7\paperheight}}{output_31_1.png}
    \end{center}
    { \hspace*{\fill} \\}
    
    \begin{Verbatim}[commandchars=\\\{\}]
Reconstruction of a sample from the first 2 eigen-vectors:
    \end{Verbatim}

    \begin{center}
    \adjustimage{max size={0.7\linewidth}{0.7\paperheight}}{output_31_3.png}
    \end{center}
    { \hspace*{\fill} \\}
    
    \begin{Verbatim}[commandchars=\\\{\}]
Reconstruction of a sample from the first 5 eigen-vectors:
    \end{Verbatim}

    \begin{center}
    \adjustimage{max size={0.7\linewidth}{0.7\paperheight}}{output_31_5.png}
    \end{center}
    { \hspace*{\fill} \\}
    
    \begin{Verbatim}[commandchars=\\\{\}]
Reconstruction of a sample from the first 10 eigen-vectors:
    \end{Verbatim}

    \begin{center}
    \adjustimage{max size={0.7\linewidth}{0.7\paperheight}}{output_31_7.png}
    \end{center}
    { \hspace*{\fill} \\}
    \pagebreak
    
    \hypertarget{digit-9}{%
\subsection{Digit 9}\label{digit-9}}

    \begin{tcolorbox}[breakable, size=fbox, boxrule=1pt, pad at break*=1mm,colback=cellbackground, colframe=cellborder]
\prompt{In}{incolor}{16}{\boxspacing}
\begin{Verbatim}[commandchars=\\\{\}]
\PY{n}{X} \PY{o}{=} \PY{n}{train\PYZus{}images\PYZus{}reshaped}\PY{p}{[}\PY{n}{train\PYZus{}labels}\PY{o}{==}\PY{l+m+mi}{9}\PY{p}{,} \PY{p}{:}\PY{p}{]}
\PY{n}{X\PYZus{}reconstruct\PYZus{}2}\PY{p}{,} \PY{n}{X\PYZus{}reconstruct\PYZus{}5}\PY{p}{,} \PY{n}{X\PYZus{}reconstruct\PYZus{}10} \PY{o}{=} \PY{n}{different\PYZus{}pcas}\PY{p}{(}\PY{n}{X}\PY{p}{,} \PY{n}{num\PYZus{}of\PYZus{}eig}\PY{p}{)}
\end{Verbatim}
\end{tcolorbox}

    \begin{Verbatim}[commandchars=\\\{\}]
The first 10 eigen-vectors:
    \end{Verbatim}

    \begin{center}
    \adjustimage{max size={0.7\linewidth}{0.7\paperheight}}{output_33_1.png}
    \end{center}
    { \hspace*{\fill} \\}
    
    \begin{Verbatim}[commandchars=\\\{\}]
Reconstruction of a sample from the first 2 eigen-vectors:
    \end{Verbatim}

    \begin{center}
    \adjustimage{max size={0.7\linewidth}{0.7\paperheight}}{output_33_3.png}
    \end{center}
    { \hspace*{\fill} \\}
    
    \begin{Verbatim}[commandchars=\\\{\}]
Reconstruction of a sample from the first 5 eigen-vectors:
    \end{Verbatim}

    \begin{center}
    \adjustimage{max size={0.7\linewidth}{0.7\paperheight}}{output_33_5.png}
    \end{center}
    { \hspace*{\fill} \\}
    
    \begin{Verbatim}[commandchars=\\\{\}]
Reconstruction of a sample from the first 10 eigen-vectors:
    \end{Verbatim}

    \begin{center}
    \adjustimage{max size={0.7\linewidth}{0.7\paperheight}}{output_33_7.png}
    \end{center}
    { \hspace*{\fill} \\}
    \pagebreak
    \hypertarget{question-2}{%
\section{Question 2}\label{question-2}}

\begin{center}\rule{0.5\linewidth}{0.5pt}\end{center}

    In the second question, we only need to know the first 20 eigen-vectors:

    \begin{tcolorbox}[breakable, size=fbox, boxrule=1pt, pad at break*=1mm,colback=cellbackground, colframe=cellborder]
\prompt{In}{incolor}{17}{\boxspacing}
\begin{Verbatim}[commandchars=\\\{\}]
\PY{n}{num\PYZus{}of\PYZus{}eig} \PY{o}{=} \PY{l+m+mi}{20}
\end{Verbatim}
\end{tcolorbox}

    \begin{tcolorbox}[breakable, size=fbox, boxrule=1pt, pad at break*=1mm,colback=cellbackground, colframe=cellborder]
\prompt{In}{incolor}{18}{\boxspacing}
\begin{Verbatim}[commandchars=\\\{\}]
\PY{n}{X} \PY{o}{=} \PY{n}{train\PYZus{}images\PYZus{}reshaped}
\PY{n}{X\PYZus{}reconstruct\PYZus{}2}\PY{p}{,} \PY{n}{X\PYZus{}reconstruct\PYZus{}5}\PY{p}{,} \PY{n}{X\PYZus{}reconstruct\PYZus{}10} \PY{o}{=} \PY{n}{different\PYZus{}pcas\PYZus{}Q2}\PY{p}{(}\PY{n}{X}\PY{p}{,} \PY{n}{num\PYZus{}of\PYZus{}eig}\PY{p}{)}
\end{Verbatim}
\end{tcolorbox}

    \begin{Verbatim}[commandchars=\\\{\}]
The first 20 eigen-vectors:
    \end{Verbatim}

    \begin{center}
    \adjustimage{max size={0.7\linewidth}{0.7\paperheight}}{output_37_1.png}
    \end{center}
    { \hspace*{\fill} \\}
    \pagebreak
    
10 samples have been selected (one from every digit group), the original sampled images is shown below:

    \begin{center}
    \adjustimage{max size={0.7\linewidth}{0.7\paperheight}}{output_37_3.png}
    \end{center}
    { \hspace*{\fill} \\}
    
    \begin{Verbatim}[commandchars=\\\{\}]
Reconstruction of samples from the first 2 eigen-vectors:
    \end{Verbatim}

    \begin{center}
    \adjustimage{max size={0.7\linewidth}{0.7\paperheight}}{output_37_5.png}
    \end{center}
    { \hspace*{\fill} \\}
    \pagebreak
    \begin{Verbatim}[commandchars=\\\{\}]
Reconstruction of samples from the first 5 eigen-vectors:
    \end{Verbatim}

    \begin{center}
    \adjustimage{max size={0.7\linewidth}{0.7\paperheight}}{output_37_7.png}
    \end{center}
    { \hspace*{\fill} \\}
    
    \begin{Verbatim}[commandchars=\\\{\}]
Reconstruction of samples from the first 10 eigen-vectors:
    \end{Verbatim}

    \begin{center}
    \adjustimage{max size={0.7\linewidth}{0.7\paperheight}}{output_37_9.png}
    \end{center}
    { \hspace*{\fill} \\}
    

    % Add a bibliography block to the postdoc
    
    
    
\end{document}
